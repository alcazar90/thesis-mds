% Template:     Tesis LaTeX
% Documento:    Archivo principal
% Versión:      3.3.1 (21/08/2023)
% Codificación: UTF-8
%
% Autor: Pablo Pizarro R.
%        pablo@ppizarror.com
%
% Manual template: [https://latex.ppizarror.com/tesis]
% Licencia MIT:    [https://opensource.org/licenses/MIT]

% CREACIÓN DEL DOCUMENTO
\documentclass[
	spanish, % Idioma: spanish, english, etc.	
	letterpaper, oneside
]{book}

% INFORMACIÓN DEL DOCUMENTO
\def\documenttitle {EXTENDING REINFORCEMENT LEARNING TECHNIQUES FOR DIFFUSION MODELS}
\def\documentsubtitle {}
\def\degreetitle {
Tesis para optar al grado de magíster en ciencia de datos
%	\bigbreak\vspace{0.3cm}
%	Memoria para optar al título de ingeniero
}

\def\universityname {Universidad de Chile}
\def\universityfaculty {Facultad de Ciencias Físicas y Matemáticas}
\def\universitydepartment {Departamento de Postgrado y Postítulo}
\def\universitydepartmentimage {departamentos/uchile2}
\def\universitydepartmentimagecfg {height=3cm}
\def\universitylocation {Santiago de Chile}

% INTEGRANTES, PROFESORES Y FECHAS
\def\documentauthor {Cristóbal Patricio Alcázar Carrasco}
\def\documentdate {\the\year}

\def\portrait {
	\begin{center}
	\vspace{1.5cm} ~ \\
	\MakeUppercase{\textbf{\documenttitle}} ~ \\
	\vspace{1.5cm}
	\MakeUppercase{\degreetitle} ~ \\
	\vfill
	\begin{tabular}{c}
		\MakeUppercase{\textbf{\documentauthor}} \\ \\
		\vspace{1.0cm} \\
		PROFESOR GUÍA: \\
		FELIPE TOBAR \\
		\vspace{0.5cm} \\
		MIEMBROS DE LA COMISIÓN: \\
		PROFESOR 2 \\
		PROFESOR 3 \\
		\vspace{0.5cm} \\
		Este trabajo ha sido parcialmente financiado por: \\
		NOMBRE INSTITUCIÓN \\
		\vspace{0.5cm} \\
		\MakeUppercase{\universitylocation} \\
		\MakeUppercase{\documentdate}
	\end{tabular}
	\end{center}
}

% abstract en ingles y español
\def\abstracttable {
	\begin{tabular}{l}
		THESIS SUMMARY TO QUALIFY FOR \\
		THE DEGREE OF MASTER OF SCIENCE \\
		IN DATA SCIENCE \\
		BY: \MakeUppercase{\documentauthor} \\
		DATE: \MakeUppercase{\documentdate} \\
		ADVISOR: FELIPE TOBAR
		%PROF. GUÍA: Felipe Tobar
	\end{tabular}
}

\def\abstracttableesp {
	\begin{tabular}{l}
		RESUMEN DE LA TESIS PARA OPTAR \\
		AL TÍTULO DE MAGÍSTER EN CIENCIAS \\
		DE DATOS \\
		POR: \MakeUppercase{\documentauthor} \\
		FECHA: \MakeUppercase{\documentdate} \\
		PROF. GUÍA: FELIPE TOBAR
	\end{tabular}
}

\def\documenttitleesp {
	EXTENSIÓN DE TÉCNICAS DE APRENDIZAJE POR REFUERZO PARA MODELOS DE DIFUSIÓN
}

% IMPORTACIÓN DEL TEMPLATE
\input{template}

% INICIO DE LAS PÁGINAS
\begin{document}

% PORTADA
\templatePortrait

% CONFIGURACIÓN DE PÁGINA Y ENCABEZADOS
\templatePagecfg

% RESUMEN O ABSTRACT
\begin{abstractd}
	Reinforcement Learning (RL) has become a pivotal tool for aligning complex generative models, addressing the limitations of traditional supervised learning methods. Its capability to optimize arbitrary rewards, including non-differentiable scalar functions or human feedback, is particularly useful for large-scale models such as Large Language Models (LLMs) and diffusion models. This thesis investigates the application of RL techniques to pretrained diffusion models, employing policy gradient methods to adapt these models for new tasks. It explores how diffusion models can be viewed as agents generating samples to maximize specific attributes—--such as aesthetic quality or compressibility--—and conducts an empirical analysis of reward signals over sample trajectories. The work includes the implementation of state-of-the-art policy optimization algorithms (DDPO) and integrates human feedback to provide practical tools and insights. This research offers a pathway for understanding the use of RL in finetuning pretrained diffusion models and provides insights for potential future adaptations.
\end{abstractd}

\begin{abstractdesp}
	El Aprendizaje por Refuerzo (RL) se ha convertido en una herramienta crucial para alinear modelos generativos complejos, superando las limitaciones de los métodos de aprendizaje supervisado tradicionales. Su capacidad para optimizar recompensas arbitrarias, incluyendo funciones escalares no diferenciables o retroalimentación humana, es especialmente útil para modelos a gran escala como los Modelos de Lenguaje Grandes (LLMs) y los modelos de difusión. Esta tesis investiga la aplicación de técnicas de RL a modelos de difusión preentrenados, utilizando métodos de gradiente de políticas para adaptar estos modelos a nuevas tareas. Explora cómo los modelos de difusión pueden considerarse agentes que generan muestras para maximizar atributos específicos, como la calidad estética o la compresibilidad, y realiza un análisis empírico de las señales de recompensa a lo largo de las trayectorias de muestra. El trabajo incluye la implementación de algoritmos de optimización de políticas de vanguardia (DDPO) e integra la retroalimentación humana para proporcionar herramientas y perspectivas prácticas. Esta investigación ofrece una ruta para comprender el uso de RL en el ajuste de modelos de difusión preentrenados y proporciona ideas para posibles adaptaciones futuras.
\end{abstractdesp}

% DEDICATORIA
\begin{dedicatory}
	A mi familia \\
	% Una frase de dedicatoria, \\
	% pueden ser dos líneas. \\
	~ \\
	% \textbf{Saludos}
\end{dedicatory}

% AGRADECIMIENTOS
%\begin{acknowledgments}
	% \ca{Debo agregar algo del fondo de investigación de la beca?}
%	\lipsum[1]
%\end{acknowledgments}

% TABLA DE CONTENIDOS - ÍNDICE
\templateIndex

% CONFIGURACIONES FINALES
\templateFinalcfg

% ======================= INICIO DEL DOCUMENTO =======================
\chapter{Introduction}

\section{Overview}

\begin{chapquote}{Morpheus, \textit{The Matrix}}
``You take the red pill, you stay in Wonderland, and I show you how deep the rabbit hole goes. Remember, all I'm offering is the truth. Nothing more.''
\end{chapquote}

\insertimage[\label{fig:anatomy-rl-algo}]{ch3-rl/anatomy-rl-algorithms.png}{scale=0.85}{Adapted from the Sergey Levine Course on Deep Reinforcement Learning.\ca{\textbf{TODO: esta imagen se va a cntextualizar dentro de modelos de difusión y reward model con RLHF.} Seguira eso si la misma idea y es un buen overview visual de la tesis}}

Reinforcement learning has shown the capacity to orchestrate or align highly complex generative models, which often proves impossible using a supervised learning objective such as matching distribution \ca{referencia ChatGPT?}. \ca{(suavizar esta transición)} Constructing agents based on generative models can be seen as a user-model interface endeavor, an intriguing line of exploration from the perspective of human-computer interaction (HCI) \ca{Agregar referencia.}. While reinforcement learning is not a cheap or intuitive approach, it offers flexibility and simplicity by optimizing a reward. Regarding the cost of sampling, highly capable generative models such as LLMs and diffusion models have fostered research efforts to reduce inference times for sample generation (inference as a first citizen). These advances make it more appealing to construct agents atop these models. \\

\noindent In this work, we propose several extensions to the formulation of the diffusion process as a sequential decision-making process, specifically regarding how to exploit the information from the intermediate state rewards rather than only using the final trajectory outcome. Based on the \textit{insights} of the reward signal behavior in sample generation, we propose methods based on the challenge classifier guidance techniques from the diffusion model literature. Moreover, we explore the use of baseline functions, a technique known to reduce the variance of the gradient estimator when using Monte Carlo estimation \cite{mohamed2020monte}, without introducing bias into the estimator. We compare the implementation of these extensions to the DDPO algorithm \cite{black2023training} on which our formulation is based, on the same \textit{downstream tasks} used in this work, such as JPEG compressibility and incompressibility, and aesthetic quality. \\

\noindent Our contributions extend the existing framework of the diffusion process by exploiting the informative intermediate state rewards rather than solely relying on the final trajectory outcome. We analyze the reward signal dynamics throughout the denoising process using a collection of sample trajectories from the \textit{google/ddpm-celebahq-256} model. Additionally, we propose extended reward functions that incorporate further information beyond the final sample, alongside the introduction of baseline functions during RL training. Model checkpoints are provided for each of the proposed methods, allowing for further exploration and comparison (see Appendix~\ref{appendix:implementation}). \\


\section{Related Work}

\noindent\textbf{Diffusion Models.} Diffusion models and score-based models represent a significant advancement in the field of generative models. These models learn a distribution $p(\mathrm{x})$ frm a dtaset $\mathcal{D}$, enabling evaluation and sampling of complex data types such as images or audio. They begin with a simple prior distribution (e.g., isotropic-Gaussian) and iteratively transform it into the target distribution through a denoising process. Recent advancements have focused on making the sampling process more efficient, as demonstrated by works like \cite{song2020denoising}, \cite{nichol2021improved}, and \cite{Salimans2022ProgressiveDF}. The progress in diffusion models has led to impressive results in tasks like text-conditional image generation \cite{ramesh2022hierarchical}, \cite{saharia2022photorealistic}, super-resolution, in-painting, style transfer, and combining different data modalities. \\

\noindent\textbf{Controling Diffusion Models.} Controlling diffusion models for new tasks is a challenging and evolving area of research. Given the high cost and resources required to train generative models from scratch, adapting pre-trained diffusion models is crucial. This adaptation allows the models to learn new concepts, such as specific objects or scenes, with minimal data. Techniques like using a small set of images to teach a model new concepts without losing its diversity \cite{ruiz2023dreambooth}, and textual inversion to embed new concepts \cite{gal2022image}, have shown promising results. Additionally, ControlNet provides advanced control over the generation process, enabling inputs like canny edges \cite{zhang2023adding}. However, many downstream tasks cannot be easily expressed through text prompts or loss functions due to their context-dependent or subjective nature. \\

\noindent\textbf{Reinforcement Learning \& Diffusion Models.} Citar trabajos de Schulman \citep{schulman2015trust} y \citep{schulman2017proximal}. The former work extend a theoretical lower bound that works for policy update, original present for the case of mixture policies (something between $\pi_{old}$ and $\pi`$), and now adapted to stochastic policies. They introduce a distance measure between policies: total variation divergence. In the following section, we will detailed more the policy optimization approach for finetuned diffusiono models with RL.\\

%Recent advancements, such as the work by [18], introduce reinforcement learning techniques to adapt pre-trained diffusion models for challenging objectives like image compressibility or prompt alignment, achieving promising results. This approach borrows ideas from reinforcement learning with human feedback (RLHF), which has been successfully used in visual language models [19]. This thesis aims to further this research by fine-tuning diffusion models through reward functions, aligning with the recent developments in RLHF and the continuous evolution of diffusion model control techniques.
\noindent\textbf{Reinforcement Learning from Human Feedback (RLHF).} Recently the attention to use human feedback in reinforcement learning has increased \cite{kaufmann2023survey}. The core idea is to capture the human feedback into a reward model that can be used to train the policy that dictates the agent's behaviour. The benefits it's to allow different types of feedback, such as binary, continuous, or even more complex signals that can be used to train in a supervised learning fashion. Then, instead of design the reward function---or use feature engineering---we can gives the agent access to the reward model to obtain the neccessary information to label trajectories and optimize its behaviour to learn the task. \\


%\noindent Diffusion \citep{sohldickstein2015deep} \citep{ho2020denoising} and score-based models \citep{song2020generative} are a kind of generative models, in which they learn from the dataset $\mathcal{D}$ a distribution $p(\mathbf{x})$ and achieve the two basic primitive operations expected for this type of models, evaluation and sampling, with the importance that scale well and handle highly complex data such as images or audio. To accomplish this task, diffusion models start with a simple prior distribution (e.g. isotropic-gaussian) and, iteratively, through a denoising process, it mutates into the target distribution. Recent years have shown an essential advance in extending and adapting the work on diffusion models, likewise making the sampling process faster in the work of \cite{song2020denoising}, \cite{nichol2021improved}, and \cite{Salimans2022ProgressiveDF}. Incorporate additional information $\mathbf{y}$ (e.g. text prompt, an image seed) to learn a conditional model $p(\mathbf{x} | \mathbf{y})$; this endeavour will be explored indirectly by guiding the sampling with an external classifier in \cite{Dhariwal2021DiffusionMB}, or achieving the same but directly in the model training phase using classifier-free guidance purpose in \cite{Ho2022ClassifierFreeDG} and effectively implemented in work \cite{nichol2022glide}. These techniques had many connections with the concept of temperature commonly used in Large Language Models (LLM) and proved helpful to guide the sampling in the gradient direction toward the space between $\mathbf{x}$ and $\mathbf{y}$ signals. Moreover, progress toward improving the likelihood estimation and exhibited impressive results on tasks such as text conditional image generation with visual language models \cite{ramesh2022hierarchical} and \cite{saharia2022photorealistic}, super-resolution \cite{ho2021cascaded}, in-painting, style transfer, and combining different data modalities. \\

%\noindent A vital research line is to use pre-trained diffusion models and adapt to novel tasks or concepts. Given the high cost of data recollection, the quantity and quality of these, and the computational resources it takes to train these "lossless" generative models from scratch. The idea is to push forward in the line that follows the LLM and their adaptation to downstream tasks \citep{Radford2019LanguageMA}. Specifically, the work of \cite{ruiz2023dreambooth} using a small number of images 3-5 of a new concept, such as a specific person, pet, or landscape, shows that a pre-trained model can learn it without sacrificing the current diversity of the model and edit the new concept or inserted it into another context (e.g. my pet surfing in Hawaii). Same goal but a different approach, \cite{gal2022image} proposed textual inversion to learn a textual embedding of the new concept, and \cite{zhang2023adding} go further with ControlNet, which not only allows you to learn a new concept with a pre-trained model but gives you additional input controls for the generation process such as canny edges. Research such as \cite{shi2023instantbooth} use adapter layers in the model architecture to reinforce the signal of the new concept without compromising the model's capabilities, in which a block is appended into the model architecture with the necessary set of weights to re-interpreted the model inference toward a signal, or more generally, a new task. However, only some downstream tasks can be easily expressed using a text prompt or a loss function. Other tasks are heavily context-dependent, challenging to define, or directly subjective. Recent work by \cite{black2023training} brings Reinforcement Learning techniques for adapting pre-trained diffusion models with challenging objectives to control via prompting, such as image compressibility or prompt alignment, and achieve promising results. Borrowing ideas from previous work and the recent development in Reinforcement Learning Human Feedback (RLHF) used by visual language models \citep{lee2023aligning}, this thesis work aims to continue this research line to fine-tuning diffusion models via rewards functions. \\

\section{Contributions and Outline}


This thesis explores the application of Reinforcement Learning (RL) techniques to adapt pre-trained diffusion models for new tasks. Specifically, it involves optimizing policies for agents built on top of the denoising network within the diffusion model framework. This approach guides the sampling process to maximize a reward signal. The main contributions of this work are: \\


\begin{enumerate}
    \item Offering the necessary background to understand the intersection of diffusion models and reinforcement learning, serving as a comprehensive guide for those looking to contribute to the field.
    \item Reproducing the study \textit{``Training Diffusion Models with Reinforcement Learning''} (Black, 2023 \cite{black2023training}), which introduces policy optimization algorithms (DDPO) for adapting pre-trained diffusion models to new tasks.
    \item Empirical Analysis of Reward Signals: Conducting empirical analyses of reward signals during the sample trajectories within the diffusion process.
    \item Presenting I-DDPO, an extended approach that considers rewards of intermediate states in the diffusion process and includes a value function to approximate a baseline.
    \item Human Feedback Integration: Demonstrating how human feedback can be incorporated into the loop via the reward model to provide guidance and align model objectives.
\end{enumerate}

\noindent To lay the groundwork, \textbf{Chapter 2} covers the basics of diffusion models, including their formulation as a Markov chain, the training objective, conditioning information or guiding model sample generation, and improvements such as speeding up sample generation. \\

\noindent Next, a broad overview of the Reinforcement Learning field is provided in \textbf{Chapter 3}, focusing on Markov Decision Processes (MDP) and Policy Optimization algorithms. The goal is to understand how to train an agent to take actions in response to its environment. \\

\ca{Afinar el párrafo del capítulo }
\noindent Equipped with the necessary background, the intersection of diffusion models and reinforcement learning is explored in \textbf{Chapter 4}. This includes reproducing the Denoising Diffusion Policy Optimization (DDPO) \cite{black2023training} using a smaller unconditional generative model and discussing the challenges and opportunities of this approach. Then, we will introduce I-DDPO that consider rewards of the intermediate state in the diffusion process and includes a value function to approoximate a baseline. Improving the monte carlo estimate of the gradient. \\

\noindent Finally, \textbf{Chapter 5} presents the conclusions of this work and future directions. \\


\chapter{Diffusion Models}

\begin{figure}[ht]
    \centering
    \includegraphics[scale=0.22]{ch2-diffusion-models/fwd_noise.png}
    \captionsetup{width=\textwidth} % set the width of the caption
    \caption{Grogu (aka Baby Yoda) joins with the Force through a forward process.}
    \label{fig:fwd-process-grogu}
\end{figure}
  
In this chapter, we introduce the Diffusion Model, a family of generative models that have proven to be a valuable framework for novel image generation, text-to-image, text-to-video, and practical applications such as molecular graph modeling and medical image reconstruction \citep{sohldickstein2015deep, ramesh2021zeroshot,rombach2022highresolution,ho2020denoising,singer2022makeavideo,jing2023torsional,song2020denoising}. Proving to be a useful tool and keep this kind of model to be an active research line in the last few years \citep{yang2024diffusion}. \\

\noindent We will review the formulation proposed in the work titled \textit{Denoising Diffusion Probabilistic Models} \cite{ho2020denoising}, or DDPM for short. The primary motivation for basing the discussion of diffusion models on this work is to understand how the pretrained model that this work is based on was trained. In addition, it serves as a framework to understand subsequent improvements and enhancements without significant modifications to the primary ingredients of the recipe. Furthermore, it has a deep connection with theoretical work of score-based generative models \citep{song2020generative} \citep{song2020improved} \citep{song2021scorebased} and Variational Autoencoders \citep{luo2022understanding}.

\section{Denoising Diffusion Probabilistic Models}

The key idea in Denoising Diffusion Probabilistic Models (DDPM) is to learn a mapping from a complex data distribution to a simple prior distribution, such as a Gaussian. This is achieved by successively corrupting the data with noise, transforming observations of the complex distribution into observations of the simple one. Concurrently, a model is trained to denoise the sequence of noisy observations and recover the original data.\\

\noindent Concretely, a forward process which starts from the raw image $\mathrm{x}_{0}$ (or whatever input) creates a sequence of intermediate steps between $\mathrm{x}_{1}$ and $\mathrm{x}_{T-1}$, also known as latent states, which are noise perturbations in some degree of the original image's structure as shown in Figure~\ref{fig:fwd-process-grogu}. At the end of the sequence $T$---\textit{if the chain large is infinitely large}---the image is ultimately converting into an isotropic Gaussian noise $p(\mathrm{x}_{T})\sim \mathcal{N}(\mathrm{0}, \mathrm{I})$\footnote{Isotropic is a fancy word for ``equal shape'', and in this context means that the direction of the covariance matrix is all equal.}.\\

\noindent The authors model this forward process $q(\mathrm{x}_{1}\dots\mathrm{x}_{T}\mid\mathrm{x}_{0})$ as a Markov chain, and it is follow a transition normal distribution without learnable parameters (aka kernel). 
\begin{align}\label{eqn:forward-process}
q(\mathbf{x}_{1:T}\mid\mathrm{x}_0) &= \prod_{t=1}^{T}q(\mathrm{x}_t\mid\mathrm{x}_{t-1})
\\
q(\mathrm{x}_t\mid\mathrm{x}_{t-1}) &= \mathcal{N}(\mathrm{x}_t;\sqrt{1-\beta_{t}}~\mathrm{x}_{t-1},~ \beta_{t}\mathrm{I})
\end{align}
A deterministic noise scheduler is known beforehand and provides the
amount of noise $\beta_{t}$ for any timestep $t$ of the Markov chain to
generate $\mathrm{x}_{t}$ . How do we think about the latent space? The scale-location transformation\footnote{A quick recap about normal distributions: if $Z$ is a standard normal random variable and $X=\mu + \sigma Z$, then $X$, then $X$ is a normal random variable with mean $\mu$ and variance $\sigma^2$, i.e., $X\sim\mathcal{N}(\mu, \sigma^2)$.} of $q(\mathrm{x}_{t}\mid\mathrm{x}_{t-1})\sim\mathcal{N}(\mathrm{x}_{t}; \sqrt{1-\beta_{t}} \mathrm{x}_{t-1}, \beta_{t}I)$ allows us to sample the \textcolor{orange}{current latent state $\mathrm{x}_{t}$ as a mixture} of the 
\textcolor{violet}{perturbation injected from noise} $\epsilon_{t-1}$ and \textcolor{teal}{the remaning structure
of the data} from the previous latent state $\mathrm{x}_{t-1}$.

%\textcolor{violet}{how much perturbation inject} from noise $\epsilon_{t-1}$, and \textcolor{teal}{the rest of what left of the image's structure} from the previous latent space $\mathrm{x}_{t-1}$:
\begin{equation}\label{eqn:scale-location-transformation}
    \textcolor{orange}{\mathrm{x}_{t}} = \textcolor{teal}{\sqrt{1-\beta_{t}}}~ \mathrm{x}_{t-1} + \textcolor{violet}{\sqrt{\beta_{t}}}~\mathrm{\epsilon}_{t-1}
\end{equation}
At the same time, there is a backward process---\textit{another Markov chain but in this case in the reverse timestep direction and with a learnable transition kernel}---model by $p_{\theta}(\mathrm{x}_{T:0})$. Specifically, 
the backward process takes the following form:
\begin{align}\label{eqn:backward-process}
    p_{\theta}(\mathrm{x}_{T:0}) &= p(\mathrm{x}_{T})\prod_{t=1}^{T}p_{\theta}(\mathrm{x}_{t-1}\mid\mathrm{x}_{t})
    \\
    p_{\theta}(\mathrm{x}_{t-1}\mid\mathrm{x_{t}}) &= \mathcal{N}(\mathrm{x}_{t-1};~\mu_{\theta}(\mathrm{x}_{t}, t),~\Sigma_{\theta}(\mathrm{x}_{t}, t))
\end{align}
We know that $p(\mathrm{x}_{T})$ is a standard normal distribution when $\beta_{T}\approx 1$ in Equation~\ref{eqn:scale-location-transformation}. Additionally, as we will see in the following sections, there are design options for learning 
the parameters of the kernel mentioned above. However, in most cases where the data distribution is complex, such as with images or audio, the kernel are parameterized by deep neural networks.\\

\section{Recursive Reparameterization Trick}\label{sec:reparameterization-trick}

% https://stats.stackexchange.com/questions/199605/how-does-the-reparameterization-trick-for-vaes-work-and-why-is-it-important
% https://web.archive.org/web/20160418040123/http://dpkingma.com/wordpress/wp-content/uploads/2015/12/talk_nips_workshop_2015.pdf
The \textit{reparameterization trick} originally appears in the work that introduce the Variational Autoencoder (VAE) model \cite{kingma2013auto} and is also used in the diffusion framework with some modifications. The main goal in the VAE context is to remove the stochasticity of a node variable that depends on parameters from the distribution, \href{https://web.archive.org/web/20160418040123/http://dpkingma.com/wordpress/wp-content/uploads/2015/12/talk_nips_workshop_2015.pdf}{leaving the stochastic part as the noise instantiated in a separate node}. As simple as it appears, this trick makes it possible to create a computational graph to properly backpropagate with automatic differentiation frameworks such as PyTorch or JAX, and update the parameters during training. Nevertheless, in the diffusion context, we will
see that a \textit{recursive} application of the reparameterization trick achieve more than just enabling backpropagation. \\

\noindent Let $\alpha_{t}=1 - \beta_{t}$ and $\bar{\alpha}_{t}=\prod_{i=1}^{t}\alpha_{i}$. Now we proceed to unwind the $t$ index until $t=0$ in Equation~\ref{eqn:scale-location-transformation}, we ended up with a reparametrization of $q(\mathrm{x}_{t}\mid\mathrm{x}_{t-1})$ in terms of $\mathrm{x}_{0}$, and it is possible to sample from it by scaling the normal standard distribution $\sim \mathcal{N}(0, \sigma^2)$ accordingly its parameters:
\begin{equation}\label{eqn:reparameterization-trick}
    \begin{split}
            \mathrm{x}_{t} & = \sqrt{\alpha_{t}}~\mathrm{x}_{t-1} + \sqrt{1-\alpha_{t}}~\epsilon_{t-1} \\
             & = \sqrt{\alpha_{t}}~(\sqrt{\alpha_{t-1}}~\mathrm{x}_{t-2} + \sqrt{1 - \alpha_{t-1}}~\epsilon_{t-2}) + \sqrt{1-\alpha_{t}}~\epsilon_{t-1} \\
             &= \sqrt{\alpha_{t}~\alpha_{t-1}}~\mathrm{x}_{t-2} + \textcolor{teal}{\sqrt{\alpha_{t} - \alpha_{t}~\alpha_{t-1}}~\epsilon_{t-2} + \sqrt{1-\alpha_{t}}~\epsilon_{t-1}} \\
             &= \sqrt{\alpha_{t}~\alpha_{t-1}}~\mathrm{x}_{t-2} + \textcolor{teal}{\sqrt{1 - \alpha_{t}~\alpha_{t-1}}~\bar{\epsilon}_{t-2}} \\
             &= \dots \\
             &= \sqrt{\alpha_{t}~\alpha_{t-1}\dots\alpha_{0}}~\mathrm{x}_{0} + \sqrt{1 - \alpha_{t}~\alpha_{t-1}\dots\alpha_{0}}~\bar{\epsilon}_{0} \\
             &= \sqrt{\prod_{i=1}^{t}\alpha_{i}}~\mathrm{x}_{0} + \sqrt{1 - \prod_{i=1}^{t}\alpha_{i}}~\bar{\epsilon}_{0} \\
             &= \sqrt{\bar{\alpha}_{t}}~\mathrm{x}_{0} + \sqrt{1 - \bar{\alpha}_
             {t}}~\bar{\epsilon}_{0} 
    \end{split}
\end{equation}
Technically, what happens in the above derivation is that the choice of a Gaussian transition kernel $q(\mathrm{x}_{t}\mid\mathrm{x}_{t-1})$ allows us to
\textcolor{teal}{group pairs of Gaussians into a single one, remaining a Gaussian distribution, by just summing up their means and variances}.  By repeteadly aplying this property, we end up marginilizing the joint distribution in Equation~\ref{eqn:forward-process} to obtain an analytical form of $q(\mathrm{x}_{t}\mid \mathrm{x}_{0})$ for all timesteps $t\in\{0, 1, \dots, T\}$ \shortcite{sohldickstein2015deep}. 
\begin{equation}\label{eqn:marginilize_q_xt_on_x0}
    q(\mathrm{x}_{t}\mid \mathrm{x}_{0}) = \mathcal{N}(\mathrm{x}_{t};\sqrt{\bar{\alpha}_{t}}~\mathrm{x}_{0}, (1-\bar{\alpha}_{t})\mathrm{I})
\end{equation}
The remarkable consequence of having a closed form is the ability to sample from $\mathrm{x}_{0}\rightarrow \mathrm{x}_{t}$ without explicitly passing through intermediate steps $\mathrm{x}_{1}, \dots, \mathrm{x}_{t-1}$. Therefore, we can go from $\mathrm{x}_{0}$ to any arbitrary $t$ in just one evaluation---\textit{since $\bar{\alpha}$ is known in advance given the noise scheduler}---making the whole DDPM framework computationally feasible.

\section{Optimization}

We can sample a batch of inputs from the data distribution $\mathrm{x}_{0}\sim q(\mathrm{x}_{0})$, such as images, and corrupt them with noise $\epsilon\sim\mathcal{N}(0, I)$ according to a scale factor $\bar{\alpha}_{t}$, obtaining the latent states from the timesteps $t\sim\mathcal{U}[1, T]$ using Equation~\ref{eqn:reparameterization-trick}. For training DDPM, the authors choose only to learn the mean of the backward kernel and use a time-dependent constant variance $\sigma_{t}^{2}$ for all $t$\footnote{Authors add a diagonal learnable variance $\Sigma_{\theta}(\mathrm{x}_{t})$ but they obtained unstable training and poor sample quality. In following work they...\ca{complete this...}}. Rather than learning the mean of the backward kernel directly, they achieve this indirectly by predicting the noise perturbation $\epsilon_{\theta}^{(t)}(\mathrm{x}_{t})$. The training objective defined in DDPM is the following:
\begin{equation}\label{eqn:ho-eq12}
    \mathbb{E}_{t\sim\mathcal{U}[1, T], \mathrm{x}_{0}\sim q(\mathrm{x}_{0}), \epsilon\sim\mathcal{N}(0, \mathrm{I})}  \big[\lambda(t) ~ \|\epsilon - \epsilon_{\theta}^{(t)}(\mathrm{x}_{t}) \|^{2} \big]
\end{equation}
Where $\lambda(t)= \beta_{t}^{2} / 2\sigma_{t}^{2}\alpha_{t}(1-\bar{\alpha}_{t})$. Empirical results shows that it is possible to discard $\lambda(t)$ without affecting sample quality, resulting in the following simplified objective:
\begin{equation}\label{eqn:ho-eq14}
    \mathbb{E}_{t\sim\mathcal{U}[1, T], \mathrm{x}_{0}\sim q(\mathrm{x}_{0}), \epsilon\sim\mathcal{N}(0, \mathrm{I})} \big[\| \epsilon - \epsilon_{\theta}^{(t)}(\sqrt{\bar{\alpha}_{t}}~\mathrm{x}_{0} + \sqrt{1-\bar{\alpha}_{t}}~\epsilon) ||^{2} \big]
\end{equation}
The simplified loss is just a mean squared error, and a description of the
training procedure is shown in Algorithm 1. Notice that the backward process $p_{\theta}(\mathrm{x}_{1:T})$ is parameterized by a collection of $T$ denoiser functions $\epsilon_{\theta}^{(t)}$ with learnable parameters $\theta^{(t)}$.
\begin{figure}[ht]\label{alg:ddpm-training-sampling}
    \begin{minipage}{0.45\textwidth}
    \begin{algorithm}[H]
        \caption{DDPM Training}
        \begin{algorithmic}
            \STATE \textbf{repeat}
            \STATE  ~~~~$\mathrm{x}_{0}\sim q(\mathrm{x_{0}})$
            \STATE  ~~~~$t\sim \mathcal{U}(1, T)$
            \STATE  ~~~~$\epsilon\sim \mathcal{N}(0, \mathrm{I})$
            \STATE  ~~~~Take gradient descent step on
            \STATE  $~~~~~~~~\nabla_{\theta}\|\epsilon - \epsilon_{\theta}^{(t)}(\sqrt{\bar{\alpha}_{t}}\mathrm{x}_{0} + \sqrt{1-\bar{\alpha}_{t}}\epsilon) \|^{2}$
            \STATE \textbf{until} converged
        \end{algorithmic}
    \end{algorithm}
    \end{minipage}
    \hspace{0.25cm}
    %\hfill
    \begin{minipage}{0.45\textwidth}
    \begin{algorithm}[H]
        \caption{DDPM Sampling}
        \begin{algorithmic}
        \STATE  $\mathrm{x}_{T}\sim\mathcal{N}(\mathrm{0}, \mathrm{I})$
        \FOR{$t=T, \dots, 1$}
            \STATE $\mathrm{z}\sim\mathcal{N}(0, \mathrm{I})$
            \STATE $\mathrm{x}_{t-1}=\frac{1}{\sqrt{\alpha_{t}}} \bigg(\mathrm{x}_{t} - \frac{1-\alpha_{t}}{\sqrt{1-\bar{\alpha}_{t}}}\epsilon_{\theta}^{(t)}(\mathrm{x}_{t})\bigg) + \sigma_{t}\mathrm{z}$
        \ENDFOR
        \STATE \textbf{return} $\mathrm{x}_{0}$
        \end{algorithmic}
    \end{algorithm}
    \end{minipage}
    \end{figure}


\noindent In the next sections we will justify the use of the above loss function and the training algorithm in the context of the Variational Lower Bound (VLB).

\subsection{Variational Lower Bound}\label{sec:variational-lower-bound}

    %https://chrisorm.github.io/VI-ELBO.html
    %esto se puede derivar desde la divergencia de Kullback-Leibler
    % o usando Jensen Inequality. En paper summary se ocupa KL, se podría
    % mostrar esa conciliaición...

DDPM are training by optimizing the Variational Lower Bound (VLB)
on the negative log-likelihood of the learnable distribution.
\begin{equation}\label{eqn:ELBO1}
    \begin{split} 
        -\log p_{\theta}(\mathrm{x}_{0}) &= -\log\int p_{\theta}(\mathrm{x}_{0:T}) d\mathrm{x}_{1:T} \\
        &= -\log \int \frac{p_{\theta}(\mathrm{x}_{0:T})q(\mathrm{x}_{1:T}\mid\mathrm{x}_{0})}{q(\mathrm{x}_{1:T}\mid\mathrm{x}_{0})}d\mathrm{x}_{1:T} \\
        &= - \log \mathbb{E}_{q(\mathrm{x}_{1:T}\mid\mathrm{x}_0)} \bigg[\frac{p_{\theta}(\mathrm{x}_{0:T})}{q(\mathrm{x}_{1:T}\mid\mathrm{x}_{0})} \bigg] \\
        % -\log p_{\theta}(\mathrm{x}_{0}) 
        & \leq \mathbb{E}_{q(\mathrm{x}_{1:T}\mid\mathrm{x}_{0})}\bigg[-\log \frac{p_{\theta}(\mathrm{x}_{0:T})}{q(\mathrm{x}_{1:T}\mid\mathrm{x}_{0})}\bigg] \\
        &\leq \mathbb{E}_{q(\mathrm{x}_{1:T}\mid\mathrm{x}_{0})}\bigg[-\log p(\mathrm{x}_{T}) - \sum_{t\geq 1}^{T} \log\frac{p_{\theta}(\mathrm{x}_{t-1}\mid\mathrm{x}_{t})}{q(\mathrm{x}_{t}\mid\mathrm{x}_{t-1})} \bigg] \\
        L &:= \mathbb{E}_{q(\mathrm{x}_{1:T}\mid\mathrm{x}_{0})}\bigg[-\log p(\mathrm{x}_{T}) - \sum_{t>1}^{T} \log \frac{p_{\theta}(\mathrm{x}_{t-1}\mid\mathrm{x}_{t})}{q(\mathrm{x}_{t}\mid\mathrm{x}_{t-1})}-\log\frac{p_{\theta}(\mathrm{x}_{0}\mid\mathrm{x}_{1})}{q(\mathrm{x}_{1}\mid\mathrm{x}_{0})}\bigg]
    \end{split}
\end{equation}

\noindent The first step in Equation~\ref{eqn:ELBO1} is to apply the definition of the negative log-likelihood and integrate out $\mathrm{x}_{1:T}$ from the
joint distribution $p_{\theta}(\mathrm{x}_{0:T})$. Then, we multiply by 1,
introducing $q(\mathrm{x}_{1:T}\mid\mathrm{x}_{0})/q(\mathrm{x}_{1:T}\mid\mathrm{x}_{0})$, and use the expectation operator. We use the Jensen's inequality to bound the negative log-likelihood by the expectation of the negative log-likelihood. \\

\noindent In the last steps, we decompose the expectation into the sum of the negative log-likelihood of the forward and backward processes, applying the Markov property to both. Notice the last term $\mathrm{x}_{T}$ is decouple from $p_{\theta}(\mathrm{x}_{0:T})$ because, for a sufficiently large sequence $T\gg$, we know that we will get a final state described by $\mathcal{N}(0, \sigma^2\mathrm{I})$. Therefore, we keep it separate in the same way as the first transition step $\mathrm{x}_{0}\mid\mathrm{x_{1}}$ and $\mathrm{x}_1\mid\mathrm{x}_0$ that involves the raw data $\mathrm{x}_{0}$ in each process.\\
    
\noindent The training objective is derived from the VLB, following the methods outlined in the relevant literature \cite{sohldickstein2015deep}\cite{ho2020denoising}\cite{luo2022understanding}. Due to space constraints, we refer to $\mathbb{E}_{q(\mathrm{x}_{1:T}\mid\mathrm{x}_{0})}$ as $\mathbb{E}_{q}$.

\begin{equation}\label{eqn:ELBO2}
    \begin{split}
        L_{\text{\scriptsize{\text{VLB}}}} & = \mathbb{E}_{q}\bigg[-\log p(\mathrm{x}_{T}) - \sum_{t>1}^{T} \log \frac{p_{\theta}(\mathrm{x}_{t-1}\mid\mathrm{x}_{t})}{q(\mathrm{x}_{t}\mid\mathrm{x}_{t-1})}-\log\frac{p_{\theta}(\mathrm{x}_{0}\mid\mathrm{x}_{1})}{q(\mathrm{x}_{1}\mid\mathrm{x}_{0})}\bigg] \\
        & = \mathbb{E}_{q}\bigg[-\log p(\mathrm{x}_{T}) - \sum_{t>1}^{T} \log \frac{p_{\theta}(\mathrm{x}_{t-1}\mid\mathrm{x}_{t})}{\textcolor{teal}{q(\mathrm{x}_{t-1}\mid\mathrm{x}_{t},~\mathrm{x}_{0})}}\frac{\textcolor{lightgray}{q(\mathrm{x}_{t-1}\mid\mathrm{x}_{0})}}{\textcolor{violet}{q(\mathrm{x}_{t}\mid\mathrm{x}_{0})}}-\log\frac{p_{\theta}(\mathrm{x}_{0}\mid\mathrm{x}_{1})}{q(\mathrm{x}_{1}\mid\mathrm{x}_{0})}\bigg] \\
        & = \mathbb{E}_{q}\bigg[-\log p(\mathrm{x}_{T}) - \sum_{t>1}^{T} \log \frac{p_{\theta}(\mathrm{x}_{t-1}\mid\mathrm{x}_{t})}{\textcolor{teal}{q(\mathrm{x}_{t-1}\mid\mathrm{x}_{t},~\mathrm{x}_{0})}} -\sum_{t>1}^{T}\log \frac{\textcolor{lightgray}{q(\mathrm{x}_{t-1}\mid\mathrm{x}_{0})}}{\textcolor{violet}{q(\mathrm{x}_{t}\mid\mathrm{x}_{0})}}-\log\frac{p_{\theta}(\mathrm{x}_{0}\mid\mathrm{x}_{1})}{q(\mathrm{x}_{1}\mid\mathrm{x}_{0})}\bigg] \\
        & = \mathbb{E}_{q}\bigg[-\log p(\mathrm{x}_{T}) - \sum_{t>1}^{T} \log \frac{p_{\theta}(\mathrm{x}_{t-1}\mid\mathrm{x}_{t})}{\textcolor{teal}{q(\mathrm{x}_{t-1}\mid\mathrm{x}_{t}, \mathrm{x}_{0})}}- \log\frac{\textcolor{orange}{q(\mathrm{x}_{1}\mid\mathrm{x}_{0})}}{\textcolor{violet}{q(\mathrm{x}_{T}\mid\mathrm{x}_{0})}} -\log\frac{p_{\theta}(\mathrm{x}_{0}\mid\mathrm{x}_{1})}{q(\mathrm{x}_{1}\mid\mathrm{x}_{0})}\bigg] \\
        & = \mathbb{E}_{q}\bigg[-\log p(\mathrm{x}_{T}) - \sum_{t>1}^{T} \log \frac{p_{\theta}(\mathrm{x}_{t-1}\mid\mathrm{x}_{t})}{\textcolor{teal}{q(\mathrm{x}_{t-1}\mid\mathrm{x}_{t}, \mathrm{x}_{0})}}- \log\textcolor{orange}{q(\mathrm{x}_{1}\mid\mathrm{x}_{0})} + \log \textcolor{violet}{q(\mathrm{x}_{T}\mid\mathrm{x}_{0})} -\log\frac{p_{\theta}(\mathrm{x}_{0}\mid\mathrm{x}_{1})}{q(\mathrm{x}_{1}\mid\mathrm{x}_{0})}\bigg] \\
        & = \mathbb{E}_{q}\bigg[-\big(\log p(\mathrm{x}_{T}) - \log \textcolor{violet}{q(\mathrm{x}_{T}\mid\mathrm{x}_{0})}\big) - \sum_{t>1}^{T} \log \frac{p_{\theta}(\mathrm{x}_{t-1}\mid\mathrm{x}_{t})}{\textcolor{teal}{q(\mathrm{x}_{t-1}\mid\mathrm{x}_{t}, \mathrm{x}_{0})}} - \big( \log\textcolor{orange}{q(\mathrm{x}_{1}\mid\mathrm{x}_{0})}  +\log\frac{p_{\theta}(\mathrm{x}_{0}\mid\mathrm{x}_{1})}{q(\mathrm{x}_{1}\mid\mathrm{x}_{0})}\big)\bigg]  \\ 
        & = \mathbb{E}_{q}\bigg[-\log \frac{p(\mathrm{x}_{T})}{\textcolor{violet}{q(\mathrm{x}_{T}\mid\mathrm{x}_{0})}} - \sum_{t>1}^{T} \log \frac{p_{\theta}(\mathrm{x}_{t-1}\mid\mathrm{x}_{t})}{\textcolor{teal}{q(\mathrm{x}_{t-1}\mid\mathrm{x}_{t}, \mathrm{x}_{0})}} -  \log p_{\theta}(\mathrm{x}_{0}\mid\mathrm{x}_{1})\bigg]  \\
        & = \mathbb{E}_{q(\mathrm{x}_{1:T}\mid\mathrm{x}_{0})}\bigg[\underbrace{D_{\text{\scriptsize{KL}}}\big(\textcolor{violet}{q(\mathrm{x}_{T}\mid\mathrm{x}_{0})} ||~p(\mathrm{x}_{T}) \big)}_{L_{T}} + \sum_{t>1}^{T} \underbrace{D_{\text{\scriptsize{KL}}}\big( \textcolor{teal}{q(\mathrm{x}_{t-1}\mid\mathrm{x}_{t}, \mathrm{x}_{0})}||~p_{\theta}(\mathrm{x}_{t-1}\mid\mathrm{x}_{t}) \big)}_{L_{t-1}} -  \underbrace{\log p_{\theta}(\mathrm{x}_{0}\mid\mathrm{x}_{1})}_{L_{0}} \bigg]
    \end{split}
\end{equation}

\noindent The key points from the above derivation are as follows: 
\begin{enumerate}
    \item Isolating  $\log q(\mathrm{x}_{t}\mid\mathrm{x}_{0})$ with $\log p(\mathrm{x}_{T})$ because both are invariant by the learning process.
    \item Reversing $q(\mathrm{x}_{t}\mid\mathrm{x}_{t-1})$ by $q(\mathrm{x}_{t-1}\mid\mathrm{x}_{t}, \mathrm{x}_{0})$ (bayes theorem) without affecting the Markov property \ca{revisar vericidad de esto...}.
    \item The first two terms at the end of Equation~\ref{eqn:ELBO2} appear due to the definition of the Kullback-Leibler divergence. 
\end{enumerate}
    

\noindent In summary, the variational lower bound loss is a sum of terms that result from the reverse diffusion process $L_{\text{\scriptsize{\text{VLB}}}}=L_{T} + L_{T-1} + \dots + L_{0}$. The final term $L_{T}$ can be discarded because
it is constant, given that $q$ has not learnable parameters and $\mathrm{x}_{T}$ follows a standard normal distribution. The remaining terms are the denoising matching terms, which are crucial in the loss function because their number depends on the length of diffusion chain $T$ (e.g. $T=1000$ in DDPM).\\

\subsection{Denoising Matching Term}

\begin{figure}[t]
  \centering
  \includegraphics[scale=0.8]{ch2-diffusion-models/DDPM-HMVAE-simple.png}
  \captionsetup{width=\textwidth} % set the width of the caption
  \caption{\textbf{Denoising matching term in action.} \textbf{Left:} $\mathrm{x}_{T}$ is a pure Gaussian noise. \textbf{Middle:} Transition from a noisy intermediate state to a less noisy one; the denoising matching term forces $p_{\theta}$ to be similar to the posterior forward kernel $q(\mathrm{x}_{t-1}\mid \mathrm{x}_{t})$. \textbf{Right:} $\mathrm{x}_{0}$ the input image during training.}
  \label{fig:ddpm-denoising-term}
\end{figure}

The model's goal is t minimize the discrepancy, measured by the Kullback-Leibler divergence, between the posterior of the encoder---\textit{which tells us how to remove the real noise}---and the decoder, which predicts how to ``reverse'' the processes and has learnable parameters $\theta$. Therefore, the loss consists of a sum of denoising matching terms across the diffusion sequence, as shown in Figure~\ref{fig:ddpm-denoising-term}. \\

\noindent However, reversing the diffusion process $q(\mathrm{x}_{t-1}\mid\mathrm{x}_{t})$ is intractable because it requires the entire dataset to compute it. We can make it tractable by conditioning on $\mathrm{x}_{0}$, leading to the following expression (See \citep{weng2021diffusion} for a detailed derivation):

\begin{align}\label{eqn:reverse-forward-process}
    q(\mathrm{x}_{t-1}\mid\mathrm{x}_{t}, \mathrm{x}_{0}) &= \mathcal{N}(\mathrm{x}_{t-1};~\tilde{\mu}_{t}(\mathrm{x}_{t}, \mathrm{x}_{0}), \tilde{\beta}_{t}I) \\
    \tilde{\mu}_{t}(\mathrm{x}_{t}, \mathrm{x}_{0}) &:= \frac{\sqrt{\bar{\alpha}_{t-1}}\beta_{t}}{1-\bar{\alpha}_{t}}\mathrm{x}_{0} + \frac{\sqrt{\alpha_{t}}(1 - \bar{\alpha}_{t-1})}{1-\bar{\alpha}_{t}}\mathrm{x}_{t} \\
    \tilde{\beta}_{t} &:= \frac{1-\bar{\alpha}_{t-1}}{1-\bar{\alpha}_{t}}\beta_{t}
\end{align}

\noindent \ca{Conectar con la intro de la sección optimization} Another important aspect is the parameterization of $p_{\theta}$ we can predict directly the latent state $\mathrm{x}_{t-1}$, or indirectly by predicting the noise perturbation $\epsilon_{\theta}$, and then apply the reparameterization trick to get the latent state, given that $\mathrm{x}_{t}$ is known. Moreover,
in (Ho et al., 2020) \cite{ho2020denoising} they use only the mean as a learnable parameter and the variance is set as $\Sigma_{\theta}(\mathrm{x}_{t}, t)=\sigma_{t}^2I$, a time dependent constant such as $\sigma_{t}^{2}=\beta_{t}$ or $\sigma_{t}^{2}=\tilde{\beta}_{t}$.
\begin{align}\label{eqn:backward-noise-reparameterization}
    \mu_{\theta}(\mathrm{x}_{t}, t) &= \frac{1}{\sqrt{\alpha_{t}}}\big(\mathrm{x}_{t} - \frac{1-\alpha_{t}}{\sqrt{1-\bar{\alpha}_{t}}}\epsilon_{\theta}(\mathrm{x}_{t}, t)\big) \\
    \mathrm{x}_{t-1} &= \mathcal{N}\big(\mathrm{x}_{t-1}; \frac{1}{\sqrt{\alpha_{t}}}\big(\mathrm{x}_{t} - \frac{1-\alpha_{t}}{\sqrt{1-\bar{\alpha}_{t}}}\epsilon_{\theta}(\mathrm{x}_{t}, t)\big), \Sigma_{\theta}(\mathrm{x}_{t}, t)\big)
\end{align}

Backward parametrerization Ho. et al just use the mean as a learnable parameter, and the variance is fixed and time-depends on the noise scheduler, 
$\Sigma_{\theta}(\mathrm{x}_{t}, t) = \sigma_{t}^{2}I$:
\begin{align}\label{eqn:backward-process-fix-variance}
    p_{\theta}(\mathrm{x}_{t-1}\mid\mathrm{x_{t}}) &= \mathcal{N}(\mathrm{x}_{t-1};~\mu_{\theta}(\mathrm{x}_{t}, t), \sigma_{t}^{2}I)
\end{align}
Therefore, we have everything to expand the denoising matching term $L_t$ with $1\leq t < T$ terms \ca{Además, emerge de la definición de KL de sumatorias entre dos gaussiano un MSE.}:

\noindent \ca{Mejorar esta parte y dar más hincapie a la VLB compartida} Finally, As you can see in Figure~\ref{fig:ddpm-denoising-term}, the denoising matching term within the gray square remarks the VAE loss term. However, there are some differences between the VAE and the DDPM: (i) VAE is optimize in the latent space, while DDPM is optimize in the data space (e.g. CelebaHQ 256x256), (ii) VAE use a kernel that is learnable, while DDPM use a fixed kernel, and (iii) VAE perform a single step of sampling, while DDPM perform a large number of steps, hence is more similar to a hierarchical VAE \citep{luo2022understanding}.\\


\subsection{Score-based generative models}\label{sec:ddpm-as-score}

% \ca{Ver sección 24.3.2 Denoising score matching (DSM) del libro de tópicos avanzados de Murphy}

In this section, we aim to establish a connection between DDPM and score-based generative models \citep{song2020generative}\citep{song2020improved}\citep{song2021scorebased}. The purpose is more conceptual than practical concerning the work of this thesis, as the literature on how to condition diffusion models is built on this theoretical perspective, and it is a way to increase the degree of control over diffusion models. \\

\noindent It turns out that the objective of Equation~\ref{eqn:ho-eq12}, derived from the variational lower bound (Section~\ref{sec:variational-lower-bound}) and the reparameterization of predicting $\epsilon_{\theta}^{(t)}$ versus directly the latent states, is equivalent to the objective function of Noise Conditional Score Network (NCSN) \citep{song2020improved}. This is a generative model that seeks to learn the data distribution through the score function. We will briefly contextualize this family of models below. \\

\noindent The score function $s(\mathrm{x}_{0})$ of the data distribution $q(\mathrm{x}_{0})$ is defined as the gradient of the log-likelihood w.r.t $\mathrm{x}_{0}$, i.e. $\nabla_{\mathrm{x}_{0}}\log q(\mathrm{x}_{0})$, and it is invariant to scale changes in the distribution. \\

\noindent First, let's start by model the data distribution $q(\mathrm{x}_{0})$ with $p_{\theta}(\mathrm{x}_{0})$:
\begin{equation}\label{eqn:model-directly-density}
    p_{\theta}(\mathrm{x}_{0}) = \frac{e^{-f_{\theta}(\mathrm{x}_{0})}}{\mathrm{Z}_{\theta}} 
\end{equation}

\noindent $f_{\theta}$ is the unnormalized probabilistic model and $\mathrm{Z}_{\theta}$ is the normalizing constant to ensure that the distribution integrates to 1.  The problem is that computing $\mathrm{Z}_{\theta}$ is intractable for high-dimensional data, but we can avoid this by inducing the score function $s_{\theta}(\mathrm{x}_{0})$ in Equation~\ref{eqn:model-directly-density}: \\

\begin{equation}\label{eqn:score-function}
    \begin{split}
        \log p_{\theta}(\mathrm{x}_{0}) &= -f_{\theta}(\mathrm{x}_{0}) - \log \mathrm{Z}_{\theta}  \\
        \nabla_{\mathrm{x}_{0}}\log p_{\theta}(\mathrm{x}_{0}) &= -\nabla_{\mathrm{x}_{0}} f_{\theta}(\mathrm{x}_{0}) - \cancel{\nabla_{\mathrm{x}_{0}}\log \mathrm{Z}_{\theta}} \\
        s_{\theta}(\mathrm{x}_{0}) &= -\nabla_{\mathrm{x}_{0}} f_{\theta}(\mathrm{x}_{0})
    \end{split}
\end{equation}

\noindent Introducción de multiples niveles de ruido $\sigma_{i}$ para lidiar con el manifold problem y obtener estimaciones del score más precisas... \\

\begin{equation}
   \mathbb{E}_{p(\mathrm{x})}\bigg[\| \nabla_{x}\log p(x) - s_{\theta}(x)\|_{2}^{2} \bigg]
\end{equation}

\begin{equation}\label{train-objective-ncsn}
    \sum_{i=1}^{L} \lambda (i) \mathbb{E}_{p_{\sigma_{i}}(\mathrm{x})} [ \nabla_{\mathrm{x}} \log p_{\sigma_{i}}(\mathrm{x}) - s_{\theta}(\mathrm{x, i}) ]
\end{equation}

\noindent \ca{paper SDE unifican los approaches como dos casos especiales} By generalizing the number of noise scales to infinity, we further proved that score-based generative models and diffusion probabilistic models can both be viewed as discretizations to stochastic differential equations determined by score functions. This work bridges both score-based generative modeling and diffusion probabilistic modeling into a unified framework.


\section{Sampling}\label{sec:sampling}

\ca{\textbf{Importante:} la reparametrización que aparece en el algoritmo de
sampling, la derivación para llegar a esta se encuentra en \href{https://lilianweng.github.io/posts/2021-07-11-diffusion-models/}{post de difusion de Lilian Weng} } \\

Once we have trained the model, $\mathrm{x}_{0}$ is obtained by sampling from the prior $p_{\theta}(\mathrm{x}_{T})$---\textit{a Gaussian distribution $\mathcal{N}(0, \mathrm{I})$}---and then applying \textit{ancestral sampling} through the reverse markovian process to get the final sample.  The Algorithm~2 shows the sampling process in detail. \\

\noindent Notice that DDPM requires training the model with a large number of steps $T$
to allow the reverse process to be close to a Gaussian. In DDPM, the hyperparameter $T$ is set to 1000 steps. One consequence of this is that sample from model is computational expensive because the ancestral sampling obtain the next generation $t-1$ from conditioning in the current state $t$ and requires to progress for every step $t$ sequentially. \\

\noindent \ca{será necesario la conexión con Langevin dynamics?} This is done by Langevin dynamics, which is a stochastic process that follows the gradient of the log-likelihood. The Langevin dynamics is defined by the following equation:
\begin{equation}
    \mathrm{x}_{t-1} = \mathrm{x}_{t} + \epsilon \nabla_{x}\log p(x) + \sqrt{2\epsilon}z_{t}, ~~~~~ t=0,1, \dots, T
\end{equation}

    
\section{Condition the model}

So far, we know how to learn indirectly the unconditional probability distribution $p(\mathrm{x})$ via the score function $\nabla_{\mathrm{x}}\log p(\mathrm{x})$, but what can we do to condition $\mathrm{x}$ given a signal $\mathrm{y}$, such as a text prompt, another image, or an audio?\\

\noindent By Bayes' theorem, log operations, and taking the gradient w.r.t. $\mathrm{x}$ we got:

\begin{equation}
    \begin{split}
        p(\mathrm{x} \mid \mathrm{y}) &= \frac{p(\mathrm{y} \mid \mathrm{x}) \cdot p(\mathrm{x})}{p(\mathrm{y})}\\
        \implies \log p(\mathrm{x} \mid \mathrm{y}) &= \log p(\mathrm{y} \mid \mathrm{x}) + \log p(\mathrm{x}) - \log p(\mathrm{y}) \\
        \implies \nabla_\mathrm{x} \log p(\mathrm{x} \mid \mathrm{y}) &= \nabla_x \log p(\mathrm{y} \mid \mathrm{x}) + \nabla_\mathrm{x} \log p(\mathrm{x}) ,
    \end{split}
\end{equation}
    
\textbf{TODO:} Review and cite the following \href{https://sander.ai/2023/08/28/geometry.html}{blog posts} \cite{dieleman2022guidance} and \cite{dieleman2023geometry} about guidance.

\subsection{Classifier Guidance (CG)}

Once a diffusion model is trained, during the sampling process (aka denoising Gaussian noise), we can attach a cost function that "guides" the denoising process toward some desired condition, such as colour or text embedding from a prompt that matches a text caption, gets by the image at each denoising step.\\ Introduce in \cite{nichol2021glide}.

\textbf{TODO:} Mention and explain the Relevant work in the last part of this work \cite{Dhariwal2021DiffusionMB}.\\     

\textbf{TODO:} Create a diagram to explain the classifier guidance. It could be used for text prompt guidance using CLIP.\\ \ca{Lo importante es mencionar como usar el espacio de embedding conjunto de CLIP en los modelos text-to-image para guairlos...}.

\begin{figure}[ht]
    \centering
    \includegraphics[scale=0.95]{ch2-diffusion-models/clip-guidance-experiment.png}
    \captionsetup{width=\textwidth} % set the width of the caption
    \caption{\textbf{CLIP classifier guidance.} Using the following prompt \texttt{old, senior, oldster, elderly, golden-ager} to denoise the noise seed that generate the four above images an update the model parameters in
    the direction of the prompt.}
    \label{fig:clip-guidance-old-experiment}
  \end{figure}

\begin{figure}[ht]
    \centering
    \includegraphics[scale=0.5]{ch2-diffusion-models/clip-overview-a.png}
    \captionsetup{width=\textwidth} % set the width of the caption
    %\caption{\textbf{CLIP overview.} Text-to-image joint embedding space. \textbf{Source:} Alec Radford, Jong Wook Kim, Chris Hallacy, Aditya Ramesh, Gabriel Goh, Sandhini Agarwal, Girish Sastry, Amanda Askell, Pamela Mishkin, Jack Clark, Gretchen Krueger, \& Ilya Sutskever. (2021). Learning Transferable Visual Models From Natural Language Supervision.}
    \caption{\textbf{CLIP overview.} Text-to-image joint embedding space. \textbf{Source:} Learning Transferable Visual Models From Natural Language Supervision \citep{radford2021learning}.}
    \label{fig:clip-overview}
  \end{figure}

  Download a pre-trained model to generate image captions such as OpenAI CLIP model
  
  \begin{enumerate}
    \item Pass the noise tensor to the model to get an encoding vector
    \item Use a loss function that compares the vector for the current sample state w.r.t. encoding for the text prompt; the last vector is always the same because the prompt doesn’t change during the process
    \item Compute the sample gradient w.r.t loss value and update the sample values in the direction that minimizes the loss
    \item Repeat the process during the whole stochastic Markovian process
  \end{enumerate}

Generally, any designed loss uses a scale factor (guidance scale $\gamma$) to increase/decrease the attribute effect. It allows you to move between novelty and fidelity. 

\subsection{Classifier Free Guidance (CFG)}

    \textbf{TODO:} A good blog post that explains classifier guidance is \cite{dieleman2022guidance}. \\


\section{Enhancements \& Improvements}

Despite the progress in conditioning the model, there are...\\

\noindent A direct improvement to the DDPM was made in \cite{nichol2021improved}, in which they learn not only $\mu$ but also the variance $\sigma^{2}$ of the reverse process. This directly improves the likelihood estimation achieved by the model. \\
% \begin{figure}[ht]
%     \centering
%     \includegraphics[scale=0.75]{ch2-diffusion-models/ddpm-linear-cosine-scheduler.png}
%     \captionsetup{width=\textwidth} % set the width of the caption
%     \caption{\textbf{Linear vs. Cosine scheduler} introduce in \cite{nichol2021improved}. Destroying the input structure smoothly is beneficial (regard what?) over using a simple linear scheduler \ca{esta figura suma realmente?}.}
%     \label{fig:ddpm-linear-vs-cosine-scheduler}
%   \end{figure}
\noindent Regarding the variance scheduler, the same work found that the linear scheduler implemented in the DDPM paper was suboptimal because destroying the image very early in the backward chain left a lot of intermediate states redundant. Instead, a cosine scheduler was proposed, which gradually destroys the image structure, providing further informative steps to the denoiser as it shown in Figure~\ref{fig:ddpm-linear-vs-cosine-scheduler}. \\

% Notably, the work of \cite{rombach2022highresolution} in improving the model architecture and the training process to scale the resolution.

\subsection{Denoising Diffusion Implicit Models}\label{sec:DDIM}
\begin{figure}[t]
  \centering
  \includegraphics[scale=0.87]{ch2-diffusion-models/DDIM-no-markovian.png}
  \captionsetup{width=\textwidth} % set the width of the caption
  \caption{\textbf{DDIM non-Markovian forward process.} \textbf{Left:} Ilustration of accelerated generation skipping the uneven intermediate steps, the denoiser network $\epsilon_{\theta}^{(t)}$ predicts the amount of noise added to $\mathrm{x}_{t}$ from step $t-2$, instead $t$. \textbf{Right:} the non-Markovian graphical model.}
  \label{fig:ddim-non-markovian}
\end{figure}

The authors demonstrate that the DDPM forward process can be reformulated as a family of non-Markovian generative processes. Their resulting variational training objectives share a common surrogate objective, precisely aligning with the one used to train DDPM. This approach allows for the use of non-Markovian diffusion processes during inference with the a pretrained DDPM model, leading to short generative Markov chains that can be sampled in just a few steps. Increasing the sample efficiency significantly but introducing a trade-off between sample quality.

\noindent Usando regla de Bayes y forzando XYZ se obtiene la siguiente expresión para el forward process que describe un proceso no-Markoviano, ilustrado en la Figura~\ref{fig:ddim-non-markovian}:
\begin{equation}\label{eqn:ddim-non-markovian-marginal-eqn7}
    q_{\sigma}(\mathrm{x}_{t-1}\mid\mathrm{x}_{t}, \mathrm{x}_{0}) = \mathcal{N}(\sqrt{\alpha_{t-1}} \mathrm{x}_0 + \sqrt{1 - \alpha_{t-1} - \sigma_{t}^{2}}\cdot\frac{\mathrm{x}_{t} - \sqrt{\alpha_{t}}\mathrm{x}_{0}}{\sqrt{1-\alpha_{t}}}, \sigma_{t}^{2}I)
\end{equation}

\noindent The backward process is...\\

\noindent \textbf{Prediction of noise-free observations.} We can obtain 
a denoised observation $\tilde{\mathrm{x}}_{t\rightarrow 0}$ from any arbitrary
intermediate state $\mathrm{x}_{t}$ by simple isolating $\mathrm{x}_{0}$ in Equation~\ref{eqn:reparameterization-trick} (Section~\ref{sec:reparameterization-trick})\footnote{In DDIM paper they use $\alpha_{t}=\alpha_{t} / \alpha_{t-1}$ to avoid using $\beta_{t}$, $\alpha_{t}$, and $\bar{\alpha}_{t}$ (See Appendix C.2 in \cite{song2020denoising})}. The caveat is that instead of sampling
 $\epsilon\sim\mathcal{N}(0, I)$,  we use the noise prediction given by the pretrained denoiser network $\epsilon_{\theta}^{(t)}$ at level $t$.

\begin{equation}\label{eqn:ddim-denoised-observation-eqn9}
    f_{\theta}^{(t)}(\mathrm{x}_{t}) = \frac{\mathrm{x}_{t} - \sqrt{1-\alpha_{t}}\epsilon_{\theta}^{(t)}(\mathrm{x}_{t})}{\sqrt{\alpha_{t}}} = \tilde{\mathrm{x}}_{t\rightarrow 0} 
\end{equation}

\noindent Then, by replacing Equation~\ref{eqn:ddim-denoised-observation-eqn9} in Equation~\ref{eqn:ddim-non-markovian-marginal-eqn7} we can compute $q_{\sigma}(\mathrm{x}_{t-1}\mid\mathrm{x}_{t}, f_{\theta}^{(t)}(\mathrm{x}_{t}))$ and sample $\mathrm{x}_{t-1}$ by:

\begin{equation}\label{eqn:ddim-sample-prev-xt-eqn12}
    \begin{split}
    \mathrm{x}_{t-1} &= \sqrt{\alpha_{t-1}}\mathrm{x}_{0} + \sqrt{1-\alpha_{t-1}-\sigma_{t}^{2}}\cdot\frac{\mathrm{x}_{t} - \sqrt{\alpha_{t}}\mathrm{x}_{0}}{\sqrt{1-\alpha_{t}}} + \sigma_{t}\epsilon_{t} \\ 
    \mathrm{x}_{t-1} &= \sqrt{\alpha_{t-1}}\bigg(\frac{\mathrm{x}_{t} - \sqrt{1 - \alpha_{t}}\epsilon_{\theta}^{(t)}(\mathrm{x}_{t})}{\sqrt{\alpha_{t}}}\bigg) + \sqrt{1 - \alpha_{t-1} - \sigma_{t}^{2}} \\ &~~~~ \cdot\frac{\textcolor{red}{\cancel{\mathrm{x}_{t}}} - \textcolor{teal}{\cancel{\sqrt{\alpha_t}}}((\textcolor{red}{\cancel{\mathrm{x}_{t}}}-\textcolor{blue}{\cancel{\sqrt{1-\alpha_{t}}}}\epsilon_{\theta}^{(t)}(\mathrm{x}_{t})) / \textcolor{teal}{\cancel{\sqrt{\alpha_{t}}}})}{\textcolor{blue}{\cancel{\sqrt{1-\alpha_{t}}}}} + \sigma_{t}\epsilon_{t} \\
    \mathrm{x}_{t-1} &= 
    \sqrt{\alpha_{t-1}}\underbrace{\bigg(\frac{\mathrm{x}_{t}-\sqrt{1-\alpha_{t}}\epsilon_{\theta}^{(t)}(\mathrm{x}_{t})}{\sqrt{\alpha_{t}}}\bigg)}_{\text{``predicted $\mathrm{x}_{0}$''}} 
    + \underbrace{\sqrt{1-\alpha_{t-1} - \sigma^2_{t}}~\cdot~\epsilon_{\theta}^{(t)}(\mathrm{x}_{t})}_{\text{``direction pointing to $\mathrm{x}_{t}$''}}
    + \underbrace{\sigma_{t}\epsilon_{t}}_{\text{random noise}}
    \end{split}
\end{equation}

\noindent \textbf{Multiple generative processes} are contain in Equation~\ref{eqn:ddim-sample-prev-xt-eqn12} depending of $\sigma$ election. Again,
the main takeaway is that all share the same surrogate objective with DDPM, so
it possible to use the same denoiser model $\epsilon_{\theta}^{(t)}$ without any additional training for different inference processes. The stochasticity comes from $\epsilon_{t}\sim\mathcal{N}(0,, I)$, a Gaussian noise independent of $\mathrm{x}_{t-1}$, and modulate by the noise scheduler. \\

\begin{enumerate}
    \item The generative process, obtained by reversing Markovian forward process in DDPM, can be derived from Equation~\ref{eqn:ddim-sample-prev-xt-eqn12} by using $\sigma_{t} = \sqrt{(1-\alpha_{t-1}) / (1 - \alpha_{t})}\sqrt{1-\alpha_{t}/\alpha_{t-1}}$ for all $t$.
    \item A deterministic forward process arise when $\sigma_{t}=0$, for all $t$, except for the initial noise. The authors names this specific generative process as denoising diffusion implicit models (DDIM).
\end{enumerate}

\noindent \textbf{Accelerate inference.} With the possiblity to design non-Markovian forward process using the same pretrained model of DDPM, the authors propose to accelerate the inference process by skipping intermediate steps. Subset of latent variables $\{\mathrm{x}_{\tau_1}, \dots, \mathrm{x}_{\tau_{s}} \}$ where $\tau$ is an increasing subset sequence of $[1, \dots, T]$ , of length $S$. We define a forward process over the subset such that $q(\mathrm{x}_{\tau_i}\mid\mathrm{x}_{0})=\mathcal{N}(\sqrt{\alpha_{\tau_{i}}}\mathrm{x}_{0}, (1-\alpha_{\tau_{i}})I)$ matches the marginals, see left part of the Figure~\ref{fig:ddim-non-markovian}. Therefore, we can accelerate the inference increasing the number of steps between the sequence subset, and introducing a trade-off between sample quality and computational cost in doing so. \\

\noindent Introducing the hyperparameter $\eta\in\mathbb{R}^{+}$ for control the
sampling stochasticity from DDIM ($\eta=0$) to DDPM ($\eta=1$) in Equation~\ref{eqn:ddim-sample-prev-xt-eqn12}, and use subsequence $\tau$ of different length 
for control the sample aceeleration.
\begin{equation}\label{eqn:ddim-to-ddpm-variance}
    \sigma_{\tau_{i}}(\eta) = \eta \sqrt{(1-\alpha_{\tau_{i-1}}) / (1-\alpha_{\tau_{i}})}\sqrt{1-\alpha_{\tau_{i}}/\alpha_{\tau_{i-1}}}
\end{equation}
It is possible to produce samples with a similar quality of DDPM $1000$ steps using $20-100$ DDIM steps by measuring the quality with the Fréchet inception distance (FID). This means a speed ups of $10\times$ to $100\times$ over DDPM generation process \cite{song2020denoising}. In practice, DDIM is commonly used as a sampler and an efficient way to accelerate the sampling process, as utilized in this thesis. \\

\noindent \textbf{Sample consistency with DDIM.} \ca{Acá hablar dela propiedaded de consistencia y explorar el espacio latente interviniendolo} As a consequence of the deterministic forward process, we have \textit{\textbf{sample consistency}}. Given any arbitrary length $\tau$, the image generated from the initial sample step $\mathrm{x}_{T}$ are fairly similar, in authors words ``it would appear that $\mathrm{x}_{T}$ alone would be an informative latent encoding of the image''.  Hence, they suggest that high level features of the images $\mathrm{x}_{0}$ are determined by the initial sample step $\mathrm{x}_{T}$. \\

\noindent The sample consistency property can be useful for inverse problems such as XYZ. For instance, in Figure~\ref{fig:ddim-inversion-pascal}, given an input image (Pedro Pascal face) we can estimate the initial noise $\tilde{\mathrm{x}}_{t}$, and run the generative process to recover the input. However, the input is not perfectly recover, the reason could be:

\begin{itemize}
    \item Interesting is that high level features are encoded by $\mathrm{x}_{T}$ and doesn't vary as long as increase the number of steps to generate the final sample. However, the quality is better as long as the the length of the trajectory increase. Therefore, a trick to obtain a recovered input with higher fidelity is to skip the first steps and starter the denoising process from then, avoiding higher feature modification.
    \item The estimation error is illustrate as a trade-off between denoising from the first step that conserve face semantic but change higher features and skipping a set of denoiser steps conserving semantic but with lower quality.
    \item External input not generated by the model are not guarantee to be recover given the generative capacities of $\epsilon_{\theta}$. In the illustration, the input is external but is the generative process is perform by a DDPM trained model on a dataset of celebrities faces.
\end{itemize}

\noindent Interesting is that high level features are encoded by $\mathrm{x}_{T}$ and doesn't vary as long as increase the number of steps to generate the final sample. However, the quality is better as long as the the length of the trajectory increase. \\

\begin{figure}[ht]
    \centering
    \includegraphics[scale=0.75]{ch2-diffusion-models/DDIM_Inversion_Pascal_2.png}
    \captionsetup{width=\textwidth} % set the width of the caption
    \caption{\textbf{DDIM inversion example using $50$ inference steps.} A Pedro Pascal photo to estimate the initial noise $\tilde{\mathrm{x}}_T$ using DDIM with the pretrained model \texttt{google/ddpm-celebahq-256} on the celebrity faces dataset CelebaHQ. Then, the input si reconstructed using the estimated noise as starter point. Below images are different results skpping the first $n$ inference steps of the denoising process.}
    \label{fig:ddim-inversion-pascal}
  \end{figure}

\section{Summary}

In summary, a typical diffusion model framework can be implemented as follows:\\

\noindent A \textbf{forward process} that consists of a Markovian chain of states $\{\mathrm{x}\}_{0:T}$ that takes the original image $\mathrm{x}_0$ and iteratively adds noise until getting isotropic Gaussian noise $\mathrm{x}_{T}$. \\

\noindent In the reverse, or \textbf{backward process}, a model learns how to gradually remove the noise to recover the data structure. Most implementations model this denoising endeavour using some U-net neural network architecture with attention modules. \\

\noindent The sequence of intermediate states provides a detailed roadmap when destroying the observation such as images and leaves a trail to the denoiser network to recover the input structure. Therefore, the forward process acts in a \textbf{self-supervised} manner generating the labels for the data itself. \\

\noindent It is possible to use a model to predict the noisy image directly, or indirectly via predicting the noise, and then substracting from the state. Nevertheless, \textbf{the core idea of the training objective is to reduce the error between the noise prediction and the true noise} used to destroy the sample at a given step. \\

\noindent The noise used to destroy the structure in data is carefully handled by a deterministic function called \textbf{noise scheduler}. It determines the variance schedule, $\beta_{1}, \dots, \beta_{T}$, or the specific amount of noise injected in each timestamp $t$ during the Markovian chain. \\

\noindent Finally, we have a model that can \textbf{generate novel samples} from the data distribution \textbf{by sampling from a standard normal distribution and then 
successively remove noise}.


\chapter{Reinforcement Learning}

Reinforcement learning (RL) \cite{Sutton1998} is all about the interaction between an agent and its environment. Learning occurs through trial-and-error, where the agent observes the state of the environment, takes actions based on these observations, influences new possible state configurations, and perceives rewards based on its previous decisions. The entire sequence of decisions is directed towards achieving a goal, such as escaping from a maze, \href{https://arxiv.org/abs/1312.5602}{winning an Atari Game} \citep{mnih2013playing}, or \href{https://deepmind.google/technologies/alphago/}{defeating the world champion of Go} \citep{silver2016mastering}. How does the agent learn to act to achieve its goal? RL algorithms are designed to maximize the total rewards obtained by the agent, thus guiding its actions towards achieving its objectives. \\

\noindent In this chapter, we will introduce RL with the essential concepts required for implementing these agents. Specifically, we will focus on model-free RL, where the agent does not aim to understand the underlying model of its environment (unlike model-based RL). Instead, the goal is to design agents that learn to perform well solely by consuming experiences from their environment. By understanding the basics of designing such agents, we will explore policy optimization methods used to learn the agent’s behavior. We will then take a step further to build agents on top of diffusion models that learn how to generate better samples aligned with the goals specified by a reward model.

\section{The Framework for Learning to Act}

The starting point for designing agents that learn to act is the Markov Decision Process (MDP) framework \cite{Sutton1998}. An MDP is a mathematical object that describes the interaction between the agent and the environment. This interaction is characterized by a tuple $\langle \mathcal{S}, \mathcal{A}, P, R, \rho_{0}, \gamma \rangle$, where:

\begin{enumerate}
    \item $\mathcal{S}$, \textbf{state space}, set of possible states in the environment.
    \item $\mathcal{A}$, \textbf{action space}, set of possible actions available to the agent.
    \item $P: \mathcal{S}\times\mathcal{A}\rightarrow\Delta(\mathcal{S})$, \textbf{transition probability distribution}, which gives the probability
    of the environment for transitioning to a new state $s_{t+1}$ with a reward $r_t$ given the current state $s_{t}$ and action $a_{t}$.
    \item $R: \mathcal{S}\times\mathcal{A}\rightarrow\mathbb{R}$, 
    \textbf{reward function}, which provides a scalar feedback signal $r_{t}$ (aka reward) $r_{t}$ to the agent after taking an action $a_{t}$ and reaching the subsequent state $s_{t+1}$.
    \item $\rho_{0}$, \textbf{initial state distribution}, which determines the probability of the agent starting in a particular state.
    \item $\gamma\in\left[0, 1 \right]$ is the \textbf{discount factor}, which determines the importance of future rewards.
\end{enumerate}

\begin{figure}[ht]
    \centering
    \includegraphics[scale=0.63]{ch3-rl/MDP-diagram.png}
    \captionsetup{width=\textwidth} % set the width of the caption
    \caption{\textbf{Left:} A loop representation of a Markov Decision Process (MDP). \textbf{Right:} An unrolled MDP depecting an episodic case with a finite horizon $T$ and a parameterized policy $\pi_{\theta}$.}
    \label{fig:mdp-diagram}
  \end{figure}


\noindent MDPs generate sequences of state-action pairs, or trajectories $\tau$, starting from an initial state $s_{0}\sim\rho_{0}$. The agent's behavior is determined by a policy $\pi:\mathcal{S}\rightarrow\Delta(\mathcal{A})$, which maps states to a distribution over actions. An action $a_{0}\sim\pi(s_{0})$ is selected, leading to the next state $s_{1}$ from transition distribution and a reward $r_{0}=R(a_{0}, s_{0})$. This process repeats iteratively, with the agent selecting actions, transitioning through states, and receiving rewards (see left side of Figure~\ref{fig:mdp-diagram}). \\

\noindent The process can run indefinitely, known as an infinite horizon, or be confined to episodes with terminal states, referred to as episodic tasks, such as winning or losing a game (see the right side of Figure~\ref{fig:mdp-diagram}). Notice that the transition depends only on the current state and action, not on the sequence of events that preceded it. This is the \textit{Markovian property}, which states that the future and the past are conditionally independent, given the present (\textit{memoryless}). In this work, we will focus on the episodic setting. \\

\noindent In reinforcement learning, the main purpose is for the agent to develop a policy $\pi$ that maximizes the expected return ($R$) for each episode. \\

\begin{equation*}
    \underset{\pi}{\text{maximize }} \mathbb{E}_{\tau\sim\pi}\left[R(\tau)\right]
\end{equation*}

\noindent The return over a trajectory $\tau$ is defined as the accumulated discounted rewards of the trajectory, $R(\tau) = \sum_{t=0}^{T-1}\gamma^{t}r_{t}$. The reward signals $r_{t}$ are the inmmediate effect of taking the actions, and the return is the cumulative rewards obtained during the trajectory, considering a discount factor $\gamma$, which gives more importance to the rewards of nearer actions than to future rewards. 


\section{Policy Optimization}

In reinforcement learning there a different approaches to solve the MDP formulated in the previous section (Figure~\ref{fig:rl-model-free-taxonomy}). The most common are value-based methods and policy-based methods. In value-based methods, the agent learns which state is more valuable and
take action that leads to it.
% a \textit{value function} ($V^{\pi}(s_{t})$) a measure goodnes of states or state-action pairs, i.e., an estimate of the expected value of a given state. 
In policy-based methods, the agent learns a policy that directly maps states to actions. This work we will focus on the last methods, specifically in policy gradients. \\

\begin{figure}[ht]
    \centering
    \includegraphics[scale=0.80]{ch3-rl/rf-solve-methods-schulman-thesis-img.png}
    \captionsetup{width=\textwidth} % set the width of the caption
    \caption{Illustration of a taxonomy of model-free RL algorithms. \textbf{Source:} \href{https://rail.eecs.berkeley.edu/deeprlcourse/}{Optimizing Expectations: From Deep Reinforcement Learning to Stochastic Coomputation Graphs} by John, Schulman (2016) \cite{schulman2016optimizing}.}
    \label{fig:rl-model-free-taxonomy}
  \end{figure}

\noindent Other approaches for finding a policy is by non solving the MDP, 
but by directly optimizing the policy. This is the case of derivative free
optimization (DFO), or evolutionary algorithms, in which the policy is parameterized by a vector $\theta$, and the agent explores the space of parameters by searching. Nothing of the temporal structure and actions of the MDPs are considered in this kind of solution. \\

% Tesis de Schulman final capítulo 4 GAE
\noindent Policy gradient methods provide a way to reduce reinforcement learning to stochastic gradient descent, by providing a connection between
how function approximation is solved in supervised learning settings.

\subsection{Learning the Policy}

The starting point is to think of trajectories as units of learning instead of individual observations (i.e., actions). What dynamics generate a trajectory? 
Given a policy $\pi_{\theta}$, represented as a function with parameter $\theta\in \mathbb{R}^{d}$, whose input is a representation of the state and whose output is action selection probabilities, we can deploy the agent into its environment at an initial state $s_0$ and observe its actions in inference mode or \textit{evaluation phase} \citep{sutton1999policy}. The agent continuously promotes actions based on the current state $s_{t}$ until the episode ends in a terminal state, when $t=T$. At this point, we can determine if the goal was accomplished, such as winning the ATARI Pong game, \textit{or generating aesthetically pleasing samples from a diffusion model}. 
% The reward signals are the inmmediate effect of taking the actions, and the returns are the cumulative rewards obtained during the trajectory. 
The returns are the scalar value that assets perfomance whether we have achieved the ultimate goal, effectively acting as a ``proxy'' of a label for the overall trajectory. Thus, the trajectory serves as our unit of learning, and the remaining task is to establish the feedback mechanism for the \textit{learning phase}. \\

\noindent Intuitivelly, we want to collect the trajectories and make the good trajectories and actions more probable, and push the actions towards betters actions. \\

\noindent Mathematically, we aim to perform stochastic optimization to learn the agent’s parameters. This involves obtaining gradient information from sample trajectories, with performance assessed by a scalar-value function (i.e., reward). The optimization is stochastic because both the agent and the environment contain elements of randomness, meaning we can only compute estimates of the gradient. Crucially, we are estimating the gradient of the expected return with respect to the policy parameters. To address this, we employ Monte Carlo Gradient Estimation \citep{mohamed2020monte}, specifically using the score function method. From a machine learning perspective, this involves dealing with the stochasticity of the gradient estimates, $\hat{g}$, and using gradient ascent algorithms to update the policy parameters based on these estimates, along with a learning rate $\alpha$ to control the step size of the optimization process.

\begin{equation}\label{eqn:gradient-ascent}
    \theta \leftarrow \theta + \alpha \hat{g}_{N}
\end{equation}


\subsection{Gradient Estimation via Score Function}\label{sec:gradient-estimation-score-function}

The gradient estimation can be obtained using the score function gradient estimator. Let's introduce the following probability objective $\mathcal{F}$, defined in the \href{https://en.wikipedia.org/wiki/Ambient_space_(mathematics)}{ambient space} $\mathcal{X}\in\mathbb{R}^n$ and with parameters $\theta\in\mathbb{R}^n$,

\begin{equation}\label{eqn:probability-objective}
\mathcal{F}(\theta) = \int_{\mathcal{X}} p(\mathrm{x; \theta})f(\mathrm{x})~d\mathrm{x} = \mathbb{E}_{p(\mathrm{x};\theta)}\big[f(\mathrm{x})\big]
\end{equation}

\noindent Here, $f$ is a scalar-valued function, similar to how the reward is represented in the reinforcement learning setting. The \textit{score function} is the derivative of the log probability distribution $\nabla_{\theta}\log p(\mathrm{x};\theta)$ with respect to its parameters $\theta$. We
can use the following identity to establish a connection between
the score function and the probability distribution $p(\mathrm{x};\theta)$.

\begin{equation}\label{eqn:log-derivative-trick-expression}
    \begin{split}
        \nabla_\theta\log p(\mathrm{x};\theta) &= \frac{\nabla_{\theta}p(\mathrm{x}; \theta)}{p(\mathrm{x};\theta)} \\
        p(\mathrm{x};\theta) \nabla_{\theta}\log p(\mathrm{x};\theta) &= \nabla_{\theta}p(\mathrm{x};\theta)
    \end{split}
\end{equation}

\noindent Therefore, taking the gradient of the objective $\mathcal{F}(\theta)$ with respect to the the parameter $\theta$, we have

\begin{equation}\label{eqn:score-function-gradient-objective}
    \begin{split}
        g = \nabla_{\theta} \mathbb{E}_{p(\mathrm{x};\theta)}[f(\mathrm{x})] &= \nabla_{\theta}\int_{\mathcal{X}} p(\mathrm{x};\theta) f(\mathrm{x}) d\mathrm{x} \\
        &= \int_\mathcal{X} \nabla_{\theta}~p(\mathrm{x}; \theta)f(\mathrm{x})d\mathrm{x} \\
        &= \int_{\mathcal{X}}p(\mathrm{x};\theta)\nabla_{\theta}\log p(\mathrm{x}; \theta) f(\mathrm{x})d\mathrm{x} \\
        &=\mathbb{E}_{p(\mathrm{x};\theta)}\big[f(\mathrm{x})\nabla_{\theta}\log p(\mathrm{x};\theta) \big] 
    \end{split}
\end{equation}

\noindent The use of the log-derivative rule of Equation~\ref{eqn:log-derivative-trick-expression} to introduce the score function in Equation~\ref{eqn:score-function-gradient-objective} is also known as the \href{https://blog.shakirm.com/2015/11/machine-learning-trick-of-the-day-5-log-derivative-trick/}{\textit{log-derivative trick}}. Now, we can compute an estimate of the gradient, $\hat{g}$, using Monte Carlo estimation with samples from the distribution $p(\mathrm{x};\theta)$ as follows:

\begin{equation}\label{eqn:score-function-gradient-estimator}
    \hat{g}_{N} = \frac{1}{N}\sum_{i=1}^{N}f\big(\hat{\mathrm{x}}^{(i)}\big) \nabla_{\theta}\log p\big(\hat{\mathrm{x}}^{(i)};\theta\big)
\end{equation}

\noindent We draw $N$ samples $\hat{\mathrm{x}}\sim p(\mathrm{x};\theta)$, compute the gradient of the log-probability for each sample, and multiply by the scalar-valued function $f$ evaluated at the sample. The average of these terms is an unbiased estimate of the gradient of the objective $g$, which we can use for gradient ascent using Equation~\ref{eqn:gradient-ascent}.

% , along with a learning rate $\alpha$ to control the step size of the optimization process. The update rule for the parameter $\theta$ is as follows:

\noindent There are two important points to mention about Equation~\ref{eqn:score-function-gradient-estimator}.

\begin{enumerate}
    \item The function $f$ can be any arbitrary function we can evaluate on $\mathrm{x}$. Even if $f$ is nondifferentiable with respect to $\theta$, it can still be used to compute the gradient estimation $\hat{g}$.
    \item The expectation of the score function is zero, meaning that the gradient estimator is unbiased.
    \begin{equation}\label{eqn:score-function-expectation-zero}
    \begin{split}
        \mathbb{E}_{p(\mathrm{x};\theta)}\big[\nabla_{\theta}\log p(\mathrm{x};\theta)\big] 
        &= \int_{\mathcal{X}}p(\mathrm{x};\theta)\nabla_{\theta}\log p(\mathrm{x}; \theta) d\mathrm{x} \\
        &= \int_{\mathcal{X}} p(\mathrm{x};\theta)\frac{\nabla_{\theta} p(\mathrm{x}; \theta)}{p(\mathrm{x};\theta)}d\mathrm{x} \\
        &= \int_{\mathcal{X}}\nabla_{\theta}p(\mathrm{x};\theta)d\mathrm{x} \\
        &= \nabla_{\theta}\int_{\mathcal{X}} p(\mathrm{x}; \theta)d\mathrm{x} = \nabla_{\theta} 1 =0
    \end{split}
    \end{equation}
\end{enumerate}

\noindent The last point is particularly useful because we can replace $f$ with a shifted version given a constant $\beta$, and still obtain an unbiased estimate of the gradient, which can be beneficial for the optimization task.

\begin{equation}\label{eqn:score-function-gradient-estimator-baseline}
\hat{g}_{N} = \mathbb{E}_{p(\mathrm{x}_{\theta})}\big[(f(\mathrm{x}) - \beta) \nabla_{\theta} \log p(\mathrm{x}; \theta)\big]
\end{equation}

\noindent Using a \textbf{\textit{baseline function}} to determine $\beta$, that does not depend on the parameter $\theta$, can reduce the variance of the 
estimator \citep{mohamed2020monte}. The baseline function, which satisfies the property in Equation~\ref{eqn:score-function-expectation-zero}, can be any function independent of
$\theta$. When a baseline is chosen to be close to the scalar-valued function $f$, it effectively reduces the variance of the estimator. This reduction in variance helps stabilize the updates by minimizing fluctuations in the gradients estimates, leading to more reliable and efficient learning.

\section{Vanilla Policy Gradient, aka REINFORCE}\label{sec:reinforce}

The REINFORCE algorithm \citep{williams1992simple} translates the previous 
derivation of gradient estimation via the score function into reinforcement learning terminology. This is the earliest member of the Policy Gradient family (Figure~\ref{fig:rl-model-free-taxonomy}), where the objective is to maximize the expected return of the trajectory $\tau$ under a policy $\pi$ parameterized by $\theta$ (e.g., a neural network). At each state $s_{t}$, the agent takes an action $a_{t}$ according to the policy $\pi$, which generates a probability distribution over actions $\pi(a_{t}\mid s_{t};\theta)$. Here, we will use the notation $\pi_{\theta}(\cdot)$ instead of $\pi(\cdot;\theta)$. \\

\noindent As we mentioned in previous section, a trajectory $\tau$ represents the sequence of state-action pairs resulting from the agent's interaction with its environment. From the initial state $s_{0}$ to the terminal state $s_{T}$, the trajectory $\tau$ is a sequence of states and actions, $\tau = (s_{0}, a_{0}, \dots, s_{T-1}, a_{T-1}, s_{T})$, which describes how the agent
acts during the episodic task. Let $p_{\theta}(\tau)$ be the
probability of obtaining the trajectory $\{\tau^{(i)}\}_{0:T}$ under the policy $\pi_{\theta}$. \\

\noindent We thus have a distribution of trajectories. Recall that the unit of learning is the trajectory $\tau$, and the goal is to maximize the expected return of the trajectory. The return $R(\tau)$ could be the cumulative rewards obtained during the \textit{episode} or the discounted rewards. The expected return is given by the following expression:

\begin{equation}\label{eqn:rl-objective}
    \mathcal{J}(\theta)=\mathbb{E}_{\tau\sim p_{\theta}(\tau)}[R(\tau)] 
\end{equation}

\noindent This is the objective we want to maximize, which is a 
particular case of Equation~\ref{eqn:probability-objective} with the
scalar-valued function $f(\mathrm{x}) = R(\tau)$, representing the return of the trajectory. Let's use the techniques from the previous section to compute the
gradient of the objective in Equation~\ref{eqn:rl-objective} with respect to the policy parameter $\theta$. The gradient estimation is given by:

\begin{equation}\label{eqn:rl-gradient-estimator-vanilla}
    \nabla_{\theta} \mathbb{E}_{\tau\sim p_{\theta}(\tau)}[R(\tau)] = \mathbb{E}_{\tau\sim p_{\theta}(\tau)}\big[R(\tau)\nabla_{\theta}\log p_{\theta}(\tau)\big]
\end{equation}    

\noindent What exactly is $p_{\theta}(\tau)$? Given that the trajectory is a sequence of states and actions, and assuming the Markov property imposed by the MDP, the probability of the trajectory is defined as follows:

\begin{equation}\label{eqn:trajectory-probability-expanded}
    \begin{split}
        p_{\theta}(\tau) &= p_\theta(s_{0}, a_{0}, s_{1}, a_{1}, \dots, s_{T-1}, a_{T-1}, s_{T}) \\
        &= \rho(s_0)~\prod_{t=0}^{T-1} \pi_{\theta}(a_{t}~|~s_{t})~P(s_{t+1}, r_{t}~|a_{t}, s_{t})
    \end{split}
\end{equation}

\noindent In the above expression, $\rho(s_{0})$ denotes the distribution of initial states, while $P(s_{t+1}, r_{t}\mid a_{t}, s_{t})$ represents the transition model, which updates the environment context based on the action $a_{t}$ taken in the current state $s_{t}$. A crucial step in estimating the gradient is computing the logarithm of the trajectory probability. Following this, we calculate the gradient with respect to the policy parameter $\theta$. 

\begin{equation}\label{eqn:trajectory-gradient-score}
    \begin{split}
        \log p_{\theta}(\tau) &= \log \rho(s_0) + \sum_{t=0}^{T-1}\log \pi_{\theta}(a_{t}~|~s_{t}) + \log P(s_{t+1}, r_{t}\mid a_{t}, s_{t}) \\
        \nabla_{\theta}\log p_{\theta}(\tau) &= \log \cancel{\nabla_{\theta}\rho(s_0)} + \sum_{t=0}^{T-1}\nabla_{\theta}\log \pi_{\theta}(a_{t}\mid s_{t}) + \log\cancel{\nabla_{\theta} P(s_{t+1}, r_{t}\mid a_{t}, s_{t})} \\
        \nabla_{\theta} \log p_{\theta}(\tau) &=  \sum_{t=0}^{T-1}\nabla_{\theta}\log \pi_{\theta}(a_{t}\mid s_{t}) 
    \end{split}
\end{equation}

\noindent The distribution of initial states and the transition
probabilities are disregarded because they are independent of $\theta$, thereby simplifying significantly the computations needed for gradient estimation. By substituting the final expression from Equation~\ref{eqn:trajectory-gradient-score} into the gradient estimation of the objective in Equation~\ref {eqn:rl-gradient-estimator-vanilla}, we derive the REINFORCE gradient estimator.

\begin{equation}\label{eqn:reinforce-gradient-estimator}
    \begin{split}
        g &= \nabla_{\theta}\mathbb{E}_{\tau\sim p_{\theta}(\tau)}[R(\tau)] \\
        &= \mathbb{E}_{\tau\sim p_{\theta}(\tau)}\left[\sum_{t=0}^{T-1} \nabla_{\theta}\log \pi_{\theta} (a_t\mid s_t) R(\tau)\right]  \\
        \hat{g} &= \frac{1}{\mid\mathcal{D}^{\pi_{\theta}}\mid}\sum_{\tau\in\mathcal{D}^{\pi_{\theta}}}\left[~\sum_{t=0}^{T-1} \nabla_{\theta} \log\pi_{\theta} (a_{t}\mid s_{t}) R(\tau) \right]
    \end{split}
\end{equation}

\noindent The core concept is to collect a set of trajectories $\mathcal{D}^{\pi_{\theta}}$ under the policy $\pi_{\theta}$ and update the policy parameters $\theta$ to increase the likelihood of high-reward trajectories while decreasing the likelihood of low-reward ones, as illustrated in Figure~\ref{fig:anatomy-rl-trajectories}. This trial-and-error learning approach, described in Algorithm~3, repeats this process over multiple iterations, reinforcing successful trajectories and discouraging unsuccessful ones, thus encoding the agent's behavior in its parameters. \\

% Basado en L3 Foundations of Deep RL series (Pieter  Abbeel)
% https://youtube.com/watch?v=AKbX1Zvo7r8
\noindent \textbf{Reduce the variance of the estimator}. Using two techniques,
\href{https://spinningup.openai.com/en/latest/spinningup/rl_intro3.html#don-t-let-the-past-distract-you}{\textit{reward-to-go}} and \textit{baseline}, we can improve the quality of the gradient estimator in Equation~\ref{eqn:reinforce-gradient-estimator}. 

\begin{figure}[ht]
    \centering
    \includegraphics[scale=0.85]{ch3-rl/simulated-trajectories-levine-slides.png}
    \captionsetup{width=\textwidth} % set the width of the caption
    \caption{\textbf{Illustration of three simulated trajectories}, denoted as $\{\tau^{(i)}\}$ where $i=(1,2,3)$, traversing the parametric space $\theta\in\mathbb{R}^2$ under the policy $\pi_{\theta}$. Each trajectory is marked with a colored symbol (cross, check) representing its \textit{goodness} based on the reward function $R(\tau^{(i)})$. \textbf{Source:} \href{https://rail.eecs.berkeley.edu/deeprlcourse/}{Policy Gradients Lecture, Deep Reinforcement Learning Course} by Sergey Levine.}
    \label{fig:anatomy-rl-trajectories}
  \end{figure}
  
% algoritmo naive REINFORCE
\begin{algorithm}
    \caption{Vanilla Policy Gradient, aka REINFORCE}
    \begin{algorithmic}
    \STATE Initialize policy $\pi_{\theta}$, set learning rate $\alpha$
    % \STATE Generate $\tau=(s_0, a_0, ..., s_{T-1}, a_{T-1}, s_{T})$ by sampling from current $\pi_{\theta}$
    \FOR {$\text{iteration}=0, 1, 2, \dots, N$}
        \STATE Collect a set of trajectories $\mathcal{D}^{\pi_{\theta}}=\{\tau^{(i)}\}$ by sampling from the current policy $\pi_{\theta}$
        \STATE Calculate the returns $R(\tau)$ for each trajectory $\tau\in\mathcal{D^{\pi_{\theta}}}$
        \STATE Update the policy: $\theta \leftarrow \theta + \alpha \bigg(\frac{1}{\mid\mathcal{D^{\pi_{\theta}}}\mid}\sum_{\tau\in\mathcal{D}^{\pi_{\theta}}}\left[\sum_{t=0}^{T-1}\nabla_{\theta}\log\pi_{\theta}(a_{t}\mid s_{t})R(\tau)\right]\bigg)$
    \ENDFOR
    \end{algorithmic}
\end{algorithm}

\noindent The reward-to-go technique is a simple trick that can reduce the variance of the gradient estimator by taking advantage of the \textit{temporal structure} of the problem. The idea is to weight the gradient of the log-probability of an action $a_{t}$ by the sum of rewards from the current timestep $t$ to the end of the trajectory $T-1$. This way, the gradient of the log-probability of an action is only weighted by the consequence of that action on the future rewards, removing terms that don't depend on $a_{t}$. Let's introduce this technique by using the gradient estimation in Equation~\ref{eqn:reinforce-gradient-estimator} and replacing $R(\tau)$ naively using the sum of total trajectory reward\footnote{The same applies for discounted returns or other kind of returns $R(\tau)$.}.

\begin{equation}\label{eqn:reinforce-gradient-reward-to-go}
    \begin{split}
        \hat{g} &= \frac{1}{\mid\mathcal{D}^{\pi_{\theta}}\mid}\sum_{\tau\in\mathcal{D}^{\pi_{\theta}}}\left[\sum_{t=0}^{T-1} \nabla_{\theta} \log\pi_{\theta} (a_{t}\mid s_{t})\sum_{t=0}^{T-1} r_{t}\right] \\
        &= \frac{1}{\mid\mathcal{D}^{\pi_{\theta}}\mid}\sum_{\tau\in\mathcal{D}^{\pi_{\theta}}}\left[~\sum_{t=0}^{T-1} \nabla_{\theta} \log\pi_{\theta} (a_{t}\mid s_{t}) \bigg( \cancel{\sum_{t=0}^{t-1} r_{t}}  + \sum_{t'=t}^{T-1} r_{t'} \bigg)\right] \\
        &= \frac{1}{\mid\mathcal{D}^{\pi_{\theta}}\mid}\sum_{\tau\in\mathcal{D}^{\pi_{\theta}}}\left[~\sum_{t=0}^{T-1} \nabla_{\theta} \log\pi_{\theta} (a_{t}\mid s_{t}) \sum_{t'=t}^{T-1} r_{t'}\right] 
    \end{split}
\end{equation}

\noindent As we see at the end of Section~\ref{sec:gradient-estimation-score-function}, it is possible to reduce the variance of the gradient estimator by using a baseline function, $b(s_{t})$, without biasing the estimator. However, is the expectation of the score still unbiased in this setting? 

\begin{equation}\label{eqn:reinforce-gradient-estimator-baseline}
    \begin{split}
        \nabla_{\theta}\mathbb{E}_{\tau\sim p_{\theta}(\tau)} &= \mathbb{E}_{\tau\sim p_{\theta}(\tau)} \bigg[\sum_{t=0}^{T-1}\nabla_{\theta}\log\pi_{\theta}(a_{t}|s_{t})  \bigg(\sum_{t'=t}^{T-1} r_{t'}-b(s_{t'}) \bigg)\bigg]
    \end{split}
\end{equation}

\noindent The proof follows a similar argument as shown in Equation~\ref{eqn:score-function-expectation-zero}, with the key difference being that the expectation is taken with respect $p_{\theta}(\tau)$, which is a sequence of random variables. By leveraging the linearity of the expectation property, we can focus on a single term at step $t$ of Equation~\ref{eqn:reinforce-gradient-estimator-baseline} to demonstrate that the baseline does not affect the expectation of the score function. We split the trajectory sequence $\tau$ at step $t$ into: $\tau_{0:t}$ and $\tau_{t+1:T-1}$, and then expand it into state-action pairs\footnote{A criterion used when splitting the trajectory is that state-action pairs are formed given that $s_{t}$ is a consequence of action $a_{t-1}$, and taking action $a_{t}$ results in state $s_{t+1}$. Notice both expectations from step 1 and 2 in Equation~\ref{eqn:reinforce-baseline-unbiased}.}.

\begin{equation}\label{eqn:reinforce-baseline-unbiased}
   \begin{split}
        \mathbb{E}_{\tau\sim p_{\theta}(\tau)}\big[\nabla_{\theta}\log\pi_{\theta}(a_t|s_t) b(s_t) \big] &=  \mathbb{E}_{\tau_{(0:t)}}\big[\mathbb{E}_{\tau_{(t+1:T-1)}}[ \nabla_{\theta}\log \pi_{\theta}(a_{t}|s_{t})b(s_{t})]\big]  \\
        &= \mathbb{E}_{s_{0:t}, a_{0:t-1}}\big[\mathbb{E}_{s_{t+1:T}, a_{t:T-1}}[ \nabla_{\theta}\log \pi_{\theta}(a_{t}|s_{t})b(s_{t})]\big] \\
        &= \mathbb{E}_{s_{0:t}, a_{0:t-1}}\big[b(s_{t})\mathbb{E}_{s_{t+1:T}, a_{t:T-1}}[ \nabla_{\theta}\log \pi_{\theta}(a_{t}|s_{t})]\big] \\
        &= \mathbb{E}_{s_{0:t}, a_{0:t-1}}\big[b(s_{t})\mathbb{E}_{a_{t}}[ \nabla_{\theta}\log \pi_{\theta}(a_{t}|s_{t})]\big] \\
        &= \mathbb{E}_{s_{0:t}, a_{0:t-1}}\big[b(s_{t})\nabla_{\theta}\mathbb{E}_{a_{t}}[\log \pi_{\theta}(a_{t}|s_{t})]\big] \\
        &= \mathbb{E}_{s_{0:t}, a_{0:t-1}}\big[b(s_{t})\nabla_{\theta}1\big] \\
        &= 0
   \end{split}
\end{equation}

\noindent We can remove irrelevant variables from the expectation over the portion of the trajectory $\tau_{(t+1):(T-1)}$ because we are focusing on the term at step $t$. The only relevant variable is $a_{t}$, and the expectation $\mathbb{E}_{a_{t}}\log\pi_{\theta}(a_{t}\mid s_{t})$ is 1. Given that the
gradient with respect to $\theta$ of a constant is zero, and $b(s_{t})$ is multiplying it, the effect of the baseline on the expectation is nullified. This argument can be applied to any other term in the sequence due to the linearity of the expectation. Therefore, we have proven that using a baseline also keeps the gradient estimator unbiased in the policy gradient setting. \\

\noindent Choosing an appropriate baseline is a critical decision in reinforcement learning \citep{foundations-deeprl-series-l3}, as different methods can offer unique strengths and limitations. Common baselines include fixed values, moving averages, and learned value functions.

\begin{enumerate}
    \item Constant baseline: $b=\mathbb{E}\left[ R(\tau)\right]\approx \frac{1}{m}\sum_{i=1}^{m} R(\tau^{(i)})$
    \item Optimal constant baseline: $b=\frac{\sum_{i}(\nabla_{\theta} \log P_{\theta}(\tau^{(i)}))^{2} R(\tau^{(i)})}{\sum_{i}(\nabla_{\theta}\log P_{\theta}(\tau^{(i)}))^{2}}$ 
    \item Time-dependent baseline: $b_{t}=\frac{1}{m} \sum_{i=1}^{m} \sum_{k=t}^{T-1} R(s_{k}^{(i)}, a_{k}^{(i)})$ 
    \item State-dependent expected return: $b(s_{t}) = \mathbb{E}\left[r_{t} + r_{t+1} + r_{t+2} + \dots + r_{T-1}\right] = V^{\pi}(s_{t})$
\end{enumerate}


\noindent The control variates method can significantly reduce estimator variance, enhancing the stability and performance of RL algorithms \cite{NIPS2001_584b98aa}. Despite the nuances and differences among baseline methods, the primary concept is the \textit{advantage}, shown in Equation~\ref{eqn:pg-objective-with-value-baseline}, which refers to increase log probabilities of action $a_{t}$ proportionally to how much its returns, $r_{t}$, are better than the expected return under the current policy, which is determined by the value function $V^{\pi}(s_{t})$.

\begin{equation}\label{eqn:pg-objective-with-value-baseline}
    \mathbb{E}_{\tau\sim p_{\theta}(\tau)} \bigg[\sum_{t=0}^{T-1}\nabla_{\theta}\log\pi_{\theta}(a_{t}|s_{t}) \bigg(\underbrace{\sum_{t'=t}^{T-1} R(a_{t'}, s_{t'}) -V^{\pi}(s_{t})}_{\text{advantage}} \bigg) \bigg]
\end{equation}

\noindent What remains is how do we get estimates for $V^{\pi}$ in practice.

\section{Actor-Critic Methods}

Actor-Critic referred to learn concurrently models for the policy and the value function. This methods are more data efficient because they amortize the samples collected $\mathcal{D}^{\pi_{\theta}}$ used for Monte Carlo estimations while reducing the variance of the gradient estimator. The actor controls how the agent behaves---\textit{by updating the policy parameters $\theta$ as we see in previous sections}---whereas the critic measures how good the taken action is, and could be a state-value ($V$) or action-value ($Q$)\footnote{Action-value function ($Q$) refers to the value of take action $a$ on state $s$ under a policy $\pi$.} function. Notice that we are combining in some way both approaches for solving MDPs as is depicted in Figure~\ref{fig:rl-model-free-taxonomy}.\\

\noindent We are introducing a new function approximator for the value function, $V_{\phi}(s_{t})$, where $\phi$ are the parameters of the value function. 

\begin{equation}\label{eqn:actor-critic-objective}
    \mathbb{E}_{\tau\sim p_{\theta}(\tau)} \bigg[\sum_{t=0}^{T-1}\nabla_{\theta}\log\pi_{\theta}(a_{t}|s_{t}) \bigg( \sum_{t'=t}^{T-1} R(a_{t'}, s_{t'}) - V_{\phi}^{\pi}(s_{t}) \bigg) \bigg]
\end{equation}

\noindent The objective is to minimize the mean squared error (MSE) between the estimated value and the empirical return, i.e. we are regress the value against empirical return in a supervised learning fashion. \ca{Mencionar conexión con el mse a partir de la varianza del gradiente? (Seita post)}:

\begin{equation}\label{eqn:value-function-loss}
   \phi \leftarrow \underset{\phi}{\arg\min} \frac{1}{\mid\mathcal{D}^{\pi_{\theta}}\mid}\sum_{\tau\in\mathcal{D}^{\pi_{\theta}}}\sum_{t=0}^{T-1}\left[\left(\left(\sum_{t'=t}^{T-1} R(a_{t'}, s_{t'})\right) - V_{\phi}(s_{t})\right)^2~\right]
\end{equation}

\noindent Algorithm~4 describes the steps for a REINFORCE variant with advantage , which combines the actor-critic approach with the traditioinoal REINFORCE algorithm. More components were introduced and can influence in the performance when the algorithm is implemented. For instance, the policy and value networks can share parameters or not. A useful study that make abalations and suggestions to pay attention when these algorithms are implemented is \textit{What Matters In On-Policy Reinforcement Learning? A Large-Scale Empirical Study (Andrychowicz, 2020 \cite{andrychowicz2020mattersonpolicyreinforcementlearning})}.\\

\begin{algorithm}
    \caption{REINFORCE with advantage}
    \begin{algorithmic}
    \STATE Initialize policy $\pi_{\theta}$
    \STATE Initialize value $V_{\phi}$
    \STATE Set learning rates $\alpha_{a}$ and $\alpha_{c}$
    \FOR {$\text{iteration}=0, 1, 2, \dots, N$}
        \STATE Collect a set of trajectories $\mathcal{D}^{\pi_{\theta}}=\{\tau^{(i)}\}$ by sampling from the current policy $\pi_{\theta}$
        \STATE Calculate the returns $R(\tau)$ for each trajectory $\tau\in\mathcal{D}^{\pi_{\theta}}$
        \STATE Update the policy:
        \STATE \qquad$\theta \leftarrow \theta + \alpha_{a} \left(\frac{1}{\mid\mathcal{D^{\pi_{\theta}}}\mid}\sum_{\tau\in\mathcal{D}^{\pi_{\theta}}}\left[\sum_{t=0}^{T-1}\nabla_{\theta}\log\pi_{\theta}(a_{t}\mid s_{t})\left(\sum_{t'=t}^{T-1} R(a_{t'}, s_{t'}) - V_{\phi}^{\pi_{\theta}}(s_{t})\right)\right]\right)$
        \STATE Update the value:
        \STATE \qquad $\phi \leftarrow \phi + \alpha_{c} \left(\frac{1}{\mid\mathcal{D}^{\pi_{\theta}}\mid}\sum_{\tau\in\mathcal{D}^{\pi_{\theta}}}\left[\sum_{t=0}^{T-1}\left(\sum_{t'=t}^{T-1} R(a_{t'}, s_{t'}) - V_{\phi}^{\pi_{\theta}}(s_{t})\right)\nabla_{\phi}V_{\phi}^{\pi_{\theta}}(s_{t})\right]\right)$
    \ENDFOR
    \end{algorithmic}
\end{algorithm}

\noindent \textbf{Advantage estimation}. Variance reduction by discounting, treateing $\gamma$ as an hyperparameter to improve $Q$ estimate...then A3C with Q k-steps lookahead estimation, and finally GAE. \ca{TODO...}


\section{Improving Sample Efficiency: Behavior and Target Policies}

% \ca{Util de referencia en literatura \cite{peshkin2002learning}}

The main drawback of the REINFORCE algorithm is its sample complexity. Once we roll out the policy and collect the data, we cannot reuse it after the policy has been updated. We must collect new data following the \textit{target policy} $\pi_{\theta}$ that we want to update. In RL literature, this is referred to as \textit{on-policy} learning. Reusing the data $\mathcal{D}\sim\pi_{\theta_{\text{old}}}$ to update the current policy $\pi_{\theta}$ would significantly improve sample efficiency\footnote{This issue also arises when attempting to transfer behavior from one task to another using existing data.}. However, once we update the policy, the previously collected data is no longer valid because the policy has changed. The distribution from which the data was sampled is now $\pi_{\theta_{\text{old}}}$. \\

\noindent Using behavior data learned from another policy, known as a \textit{behavior policy}, to update the current policy is referred to as \textit{off-policy} learning in RL literature. Let's introduce a \textit{behavior policy} in the RL objective defined in Equation~\ref{eqn:rl-objective} using \href{https://timvieira.github.io/blog/post/2014/12/21/importance-sampling/}{importance sampling} (See Mckay book, Section 29.2 \cite{mackay-book}):

% Derive RL objective with importance sampling to use data from another policy
\begin{equation}\label{eqn:derive-rl-objective-with-is}
    \begin{split}
        \nabla_{\theta}\mathcal{J}(\theta) &= \mathbb{E}_{\tau \sim p_{\theta}(\tau)}\bigg[ \nabla_{\theta}\log p_{\theta}(\tau) R(\tau)\bigg] \\
        &= \mathbb{E}_{\tau \sim p_{\theta}(\tau)} \bigg[ \frac{\nabla_{\theta}p_{\theta}(\tau)}{p_{\theta}(\tau)} R(\tau) \bigg] \\
        &= \int_{\mathcal{X}} p_{\theta}(\tau) \frac{\nabla_{\theta}p_{\theta}(\tau)}{p_{\theta}(\tau)} R(\tau) d\tau \\
        &= \int_{\mathcal{X}} \frac{p_{\theta_{\text{old}}}(\tau)}{p_{\theta_{\text{old}}}(\tau)} \cancel{p_{\theta}(\tau)} \frac{\nabla_{\theta}p_{\theta}(\tau)}{\cancel{p_{\theta}}(\tau)} R(\tau) d\tau \\
        &= \int_{\mathcal{X}} p_{\theta_{\text{old}}}(\tau) \frac{\nabla_{\theta}p_{\theta}(\tau)}{p_{\theta_{\text{old}}}(\tau)} R(\tau) d\tau \\
        &= \mathbb{E}_{\tau\sim p_{\theta_{\text{old}}}(\tau)} \bigg[\frac{\nabla_{\theta} p_{\theta}(\tau)}{p_{\theta_{\text{old}}}(\tau)} R(\tau)\bigg]
    \end{split}
\end{equation}

\noindent We derive a new objective that is more general and reconciles both \textit{on-policy} and \textit{off-policy} learning in the importance weight,
or importance correction ($p_{\theta}(\tau) / p_{\theta_{\text{old}}}(\tau)$). 

% RL objective with IS
\begin{equation}\label{eqn:rl-objective-with-is}
    \mathcal{J}_{\text{IS}}(\theta) = \mathbb{E}_{\tau\sim p_{\theta_{\text{old}}}(\tau)}\bigg[\frac{p_{\theta}(\tau)}{p_{\theta_{\text{old}}}(\tau)} R(\tau)\bigg]
\end{equation}

\noindent We can assume that the data collected from the behavior policy is
not so different from the target policy, and use first order approximation to
update the policy. 

\begin{equation}\label{eqn:rl-objective-is-linear-aprox}
    \begin{split}
        \nabla_{\theta}\mathcal{J}(\theta)\rvert_{\theta=\theta_{\text{old}}} &= \mathbb{E}_{\tau\sim p_{\theta_{\text{old}}}(\tau)} \bigg[\frac{\nabla_{\theta} p_{\theta}(\tau)\rvert_{\theta=\theta_{\text{old}}}}{p_{\theta_{\text{old}}}(\tau)} R(\tau)\bigg] \\
        &= \mathbb{E}_{\tau\sim p_{\theta_{\text{old}}}(\tau)} \big[\nabla_{\theta}\log p_{\theta}(\tau)\rvert_{\theta=\theta_{\text{old}}} R(\tau) \big]
    \end{split}
\end{equation}


\noindent \textbf{The problem with first order approximation}. The gradient estimation it is good only in the inmediate vecinity, because is a local approximation of the function. Hence, the step size is crucial to avoid a policy degradation, a situation where the policy is updated with a bad gradient,
it is difficult to recover from this situation. Given that the data is collected by the policy, the feedback loop can be dangerous for the training
stability. \\

\section{Trust Region and Proximal Policy Optimization}\label{sec:trpo-ppo}

Trust Region Policy Optimization (TRPO) \cite{schulman2015trust} allows us to
avoid the policy degradation given bad updates. The idea is to 
define a trust region in which update the policy parameter is safer and
balancing the policy improvement with stability.

\begin{align}
    \text{Surrogate loss:} \quad & \underset{\pi_{\theta}}{\max}~L(\pi) = \mathbb{E}_{\pi_{\theta_{\text{old}}}} \left[ \frac{\pi_{\theta}(a\mid s)}{\pi_{\theta_{\text{old}}}(a\mid s)} A^{\pi_{\theta_{\text{old}}}}(s, a) \right] \label{eqn:trpo-loss} \\
    \text{Constraint:} \quad & \mathbb{E}_{\pi_{\text{old}}} \left[ D_{\text{KL}}(\pi_{\theta} || \pi_{\theta_{\text{old}}}) \right]  \leq \epsilon, \nonumber
\end{align}

%\noindent \textbf{Maximize data efficiency in comparison to traditional policy gradients}. 

\noindent Increase data efficiency while avoiding step size problems in updating parameters, compared to traditional policy gradients (PG). The main idea is to improve a surrogate objective significantly while making minimal changes to the policy. These minimal changes are quantified using the KL divergence between action distributions. The trust region is the area where the new policy remains close to the old one, guarantee training stability. \\

% The trust region is the area where the new policy remains close to the old one, allowing for constrained improvement. \\

\noindent Proximal Policy Optimization (PPO) \cite{schulman2017proximal} is about simplify TRPO in order to (i) be easier to implement avoiding solve the second order optimization in Equation~\ref{eqn:trpo-loss}, (ii) taking advantage of first order optimizer such as ADAM \cite{kingma2017adammethodstochasticoptimization}, and (iii) be more compatible with neural networks operations such as dropout that are incompatible with TRPO setting\footnote{Which use the stochastic graph with hessian...}. \\

\noindent Let's rename the importance weights as the probability ratio $r$: 

\begin{equation}\label{eqn:importance-ratio-is}
    r_{t}(\theta) = \frac{\pi_{\theta}(a_{t}\mid s_{t})}{\pi_{\theta_{\text{old}}}(a_{t}\mid s_{t})}
\end{equation}

\noindent The strategy is to keep this ratio closer to 1. We can create a trust region via clipping the ratio to force within a range $\left[1-\epsilon, 1+\epsilon \right]$. \ca{\textbf{TODO:} Agregar las dos versiones de funciones de loss en paper PPO.}

\begin{equation}\label{eqn:clip-ac-objective}
\mathcal{L}^{\text{CLIP}}(\theta) = \hat{\mathbb{E}}_t \left[ \min \left( r_t(\theta) \hat{A}_t, \text{clip}(r_t(\theta), 1 - \epsilon, 1 + \epsilon) \hat{A}_t \right) \right]
\end{equation}

\noindent For a walkthrough implementation that cover important details avoid in the paper and that impact significatnly in the performance, review (Huang, 2023 \cite{dlr191986}).


% algoritmo naive REINFORCE
\begin{algorithm}
    \caption{Proximal Policy Optimization (PPO), Actor-Critic Style}
    \begin{algorithmic}
    \STATE Initialize policy parameter $\theta$, set learning rate $\alpha$
    \STATE Initialize value $V_{\phi}$
    \FOR{$\text{iteration}=0, 1, 2, \dots N$}
        \FOR{$\text{actor}=0, 1, 2, \dots M$}
            \STATE Run policy $\pi_{\theta_{\text{old}}}$ in environment for $T$ timesteps        
            \STATE Compute advantage estimates $\hat{A}_{0}, \dots, \hat{A}_{T-1}$
        \ENDFOR
        \STATE Optimize surrogate $\mathcal{L}^{\text{CLIP}}$ wrt $\theta$ (Equation~\ref{eqn:clip-ac-objective}), with $K$ epochs and minibatch size $M\leq NT$
    \ENDFOR
    \end{algorithmic}
\end{algorithm}


\section{Reinforcement Learning From Human Feedback}\label{sec:rlhf}

Reinforcement learning from human feedback (RLHF) is introducing the human within the reinforcement loop, providing the agent the necessary feedback to learn the intended behavior. The idea is learning a reward model that capture the behavior from humans, and use the model to provide the feedback in asynchrounous way, which means that during agent training it is not require to wait the human for a respond with the reward\footnote{A static reward model is equivalent to extract information in a off-policy fashion. But, the reward model can update dynamically with the agent, taking new information from the on-policy trajectoriesa.}; making fully compatible and scalable in the model-free and online learning setting studied in this chapter. \\

\noindent A reward function elicit changes in the system behavior, that is the whole point of reinforcement learning. By trial-and-error, figuring out what is the best behavior that maximize the reward, and by that means, the intended goal. However, the reward function is a key component in the RL setting and is not always easy to design it. Outside video games with clear rules and scenarios, real world is highly complex, erratic, and dificult to simulate. Therefore, more than sometimes align the intended behavior with the reward function is a difficult task. Beyond the realm of RL, use human feedback in this setting provide a useful way to align matching distribution learning systems with the human preferences. Perhaps, the most significant example is in large language models, in which RLHF can mute the behavior of an inherent autocompletion system into a more instructional and conversationl behavior \cite{ouyang2022training}.

\begin{figure}[ht]
    \centering
    \includegraphics[scale=0.55]{ch3-rl/rl-standard-vs-rlhf-settings.png}
    \captionsetup{width=\textwidth} % set the width of the caption
    \caption{\textbf{Left:} A standard RL settings. \textbf{Right:} A RLHF setting considering reward modeling. \textbf{Source:} \href{https://arxiv.org/abs/2312.14925}{A Survey of Reinforcement Learning from Human Feedback (Kaufmann et al., 2024) \cite{kaufmann2023survey}}. Notice how the reward model is decoupled from the environment and the relation highlighted between the reward model and an oracle (i.e. labeler) who provides a label to a given query.}
    \label{fig:rlhf-setting}
  \end{figure}

\noindent \textbf{Training the reward model}. Translate the human feedback 
from a collection of trajectory samples into a reward signal. However,
the human feedback can be noisy, inconsistent, and sparse. So, instead of
using the human feedback directly, it is used a reward model that learns
via an utility function that design to consume different types of feedbacks. For instance, a common setting is a binary comparison between trajectories where a utilify function is learned from the human preference between both using the Bradley-Terry model \cite{bradley-terry-model}. \\

\begin{equation*}
P(\tau_{1}\succ\tau_{2}) = \frac{1}{1 + \exp(R(\tau_{2}) - R(\tau_{1}))}
\end{equation*}

\noindent given a collection of trajectories $\mathcal{D}$ where $\succ$ means ``preferred to'' and $R(\tau)$ correponds to the utility (i.e. return in the context of RL). ...it could be between an intermediate feature maps that allows to the human gives feedback in a useful way but provide a scalar signal easily to digest by the agent. The reward model can be trained using supervised learning, imitation learning, or inverse reinforcement learning. The reward model can be used in different ways, such as reward shaping, reward augmentation, or reward correction. \\

\begin{equation}
    \underset{\phi}{\max} \prod_{i=1}^{N}\frac{1}{1 + \exp\left(1 + \exp(R_{\Phi}) \right)}
\end{equation}

% \noindent \textbf{Alignment via reward modeling}. Train a reward model over human preferences \cite{leike2018scalable}, then use it to train the agent. The model can provide a reward signal that is more aligned with the human preferences. \\

\noindent Several human feedback types can distill into rewards models such as critique, comparisons, inter-temporal feedback, proxy rewards, social behavior, improvements, and natural language \cite{kaufmann2023survey}. Specifically,
for visual perception tasks, some interesting research lines are leveraging general human knowledge from large pretrained visual language models to provide feedback to the agent if it achieve success (i.e. success detectors) on the task in which is reinforced \cite{du2023visionlanguagemodelssuccessdetectors}. In the context of diffusion models \cite{black2023training}, a language vision for the model (e.g. LLaVA) allows into a joint-embedding space and compute a BERT score... model (e.g. LLaVA) for prompt alignment. The idea is to give samples generated. There is also the work that \cite{lee2023aligning}.


\section{Summary}

In this chapter, we have explored the foundational concepts and methodologies in reinforcement learning (RL). The core of RL is the interaction between an agent and its environment, where learning occurs through trial-and-error. The agent's goal is to maximize cumulative rewards by taking actions based on its observations, influencing the state of the environment, and receiving rewards. \\

\noindent We began by introducing the Markov Decision Process (MDP), a mathematical framework that describes the interaction between the agent and the environment. An MDP is characterized by a state space, action space, transition probabilities, and reward functions. The agent aims to learn a policy that maximizes the expected return, which is the sum of discounted rewards over time. \\

\noindent We then delved into policy optimization methods, focusing on policy gradients, a popular approach in model-free RL. Policy gradient methods reduce RL to a problem of stochastic gradient descent, leveraging trajectories of state-action pairs to update the policy parameters. We discussed techniques such as the reward-to-go and baselines to reduce the variance of gradient estimators, thus improving learning efficiency. \\

\noindent Additionally, we examined how RL can be integrated with diffusion models to enhance sample generation. By training a reward model to align generated samples with desired goals, agents can learn to produce high-quality outputs in tasks such as image generation and other creative applications. \\

\noindent In conclusion, reinforcement learning offers a powerful framework for designing intelligent agents capable of learning optimal behaviors through interaction with their environment. By understanding and implementing the principles and techniques covered in this chapter, one can develop sophisticated RL agents for a wide range of applications \\

\chapter{Extending Reinforcement Learning in Diffusion Models}

The code repository \href{https://github.com/alcazar90/ddpo-celebahq}{https://github.com/alcazar90/ddpo-celebahq}.

\section{Introduction}

\textbf{TODO:} Introducir los modelos de difusión y luego la intersección de
estos con reinforcement learning. \\

Reinforcement learning has shown the capacity to orchestrate or align highly complex generative models, which often proves intractable using supervised learning matching distribution objective. Constructing agents based on generative models can be seen as an user-model interface, an intriguing line of exploration from the perspective of human-computer interaction (HCI). While reinforcement learning is not a cheap or intuitive approach, it offers flexibility and simplicity by optimizing a reward. Regarding the cost of sampling, highly capable generative models such as LLMs and diffusion models have fostered research efforts to reduce inference times for sample generation (inference as a first citizen). These advances make it more appealing to construct agents atop these models. \\

In this work, we propose several extensions to the formulation of the diffusion process as a sequential decision-making process, specifically regarding how to exploit the information from the intermediate state rewards rather than only using the final trajectory outcome. Based on the \textit{insights} of the reward signal behavior in sample generation, we propose methods based on the challenge classifier guidance techniques from the diffusion model literature. Moreover, we explore the use of baseline functions, a technique known to reduce the variance of the gradient estimator when using Monte Carlo estimation \cite{mohamed2020monte}, without introducing bias into the estimator. We compare the implementation of these extensions to the DDPO algorithm \cite{black2023training} on which our formulation is based, on the same \textit{downstream tasks} used in this work, such as JPEG compressibility, JPEG incompressibility, and aesthetic quality as proposed in \cite{black2023training}. \\

Our contributions extend the existing framework of the diffusion process by exploiting the informative intermediate state rewards rather than solely relying on the final trajectory outcome. We analyze the reward signal dynamics throughout the denoising process using a collection of sample trajectories from the \textit{google/ddpm-celebahq-256} model. Additionally, we propose extended reward functions that incorporate further information beyond the final sample, alongside the introduction of baseline functions during RL training. We compare the implementation of these extensions with the DDPO algorithm, highlighting the advantages of our framework in downstream tasks such as JPEG compressibility, JPEG incompressibility, and aesthetic quality assessment.

\section{Related Work}

\textbf{TODO:} Esta sección tiene que ser escrita muy en modo paper Neurips
para reciclar lo máximo posible. Explicar el trabajo de Black y otros. \\


Introducir qué es Reinforcement Learning with Human Feedback. Vincular con el paper de Ouyang en modelos de lenguaje: Training language models to follow instructions with human feedback. Ver tambien el substack de Nathan Lambert (https://www.interconnects.ai/p/specifying-objectives-in-rlhf). 

Papers relevantes para establecer relación entre modelos de difusión y RL.

\begin{itemize}
    \item Training Diffusion Models with Reinforcement Learning (Black 2023)
    \item Aligning Text-to-Image Models using Human Feedback (Lee 2023). Interesting discussion about how to train the reward function, this work provides pseudo-code with the steps used for training. However, the apporach purposed in this work is off-line and doesn't generate samples to update the model using a policy optimization algorithm like DDPO.
    \item Trust Region Policy Optimization \& Proximal Policy Optimization Algorithms (Schulman 2015 and 2017). The former work extend a theoretical lower bound that works for policy update, original present for the case of mixture policies (something between $\pi_{old}$ and $\pi`$), and now adapted to stochastic policies. They introduce a distance measure between policies: total variation divergence. 
\end{itemize}

\section{Background: Diffusion Model as Sequential Decision-making Process}

Several works postulate probabilistic models as MDPs...

% Diffusion Model as MDP
\begin{figure}[ht]
  \centering
  \includegraphics[scale=0.85]{img/results/diffusion-model-MDP.png}
  \vspace{-4pt}  % reduce space between caption and figure
    \captionsetup{width=\textwidth} % set the width of the caption
    \caption{\textbf{Diffusion model as a sequential decision-making process.} The policy
  $\pi_{\theta}:=p_{\theta}(\mathrm{x}_{t-1} | \mathrm{x}_{t})$ start taking denoising decisions from pure noise until the final sample $\mathrm{x}_0$, through the entire backward process $\tau=\mathrm{x}_{T:0}$. \textbf{TODO:} mejorar este caption...}
  \label{fig:diffusion-model-mdp}
\end{figure}

% agregar función de reward...
We consider the denoising diffusion policy optimization (DDPO) 
\cite{black2023training} formulation as starting point. The initial state
$\mathrm{s}_{0}$ is sampling from an isotropic Gaussian distribution
 $\rho_{0}\sim\mathcal{N}(0, I)$, corresponding to the noise
 $\mathrm{x}_{T}$ at the beginning of the diffusion backward process (Figure~\ref{fig:diffusion-model-mdp}), when the
sampling process start. The denoising neural network $p_{\theta}$ using to
estimate $\mathrm{x}_{t-1}$ at each timestep $t$ directly, or indirectly by
estimate the noise $\hat{\epsilon}_{t}$, is treat as the policy
$\pi_{\theta}(a_{t}, s_{t})$, in which the agent via its denoising action moves 
from a noisy step to a less noisy one, i.e. $a_t: x_{t} \rightarrow x_{t-1}$.

In this framework, we can optimize the diffusion model parameters 
$\theta$ directly via policy gradient estimation to maximize any arbitrary
scalar-reward signal over the sample $\mathrm{x}_{0}$ producing by the diffusion
process, using the following objective:
\begin{equation}\label{difusion-rl-objective-1}
  \mathcal{J}_{\text{DDRL}}(\theta)
  = \mathbb{E}_{\mathrm{c}\sim p(\mathrm{c}),  \mathrm{x}_{0}\sim p_{\theta}(\mathrm{x}_{0}|\mathrm{c})}[ r(\mathrm{x}_{0}, \mathrm{c})]
\end{equation}
An important point of this formulation is that the reward $R(s_{t}, a_{t})$
only provide information from the final sample $\mathrm{x}_{0}$, giving zero
reward to every non-terminal state, or $\mathrm{x}_{t}$ where $t\neq0$ as it is
depicted in Figure~\ref{fig:diffusion-model-mdp}.

Given the objective of this agent, the DDPO works presents two objective 
variations, the traditional REINFORCE objective \cite{williams1992simple} called $\text{DDPO}_{\text{SF}}$, where SF stand by score function method to estimate gradients from samples using Monte Carlo \cite{mohamed2020monte}, and $\text{DDPO}_{\text{IS}}$ that used importance sampling to optimize a surrogate objective \cite{schulman2015trust} \cite{schulman2017proximal}.
% agregar objetivo DDPO_{SF}
\begin{equation}\label{eqn:ddpo-sf-objective}
  (\text{DDPO}_{\text{SF}})~~ \nabla_{\theta}\mathcal{J} = \mathbb{E}_{\mathrm{x}_{T:0}\sim p_{\theta}} \bigg[\sum_{t=0}^{T}\nabla_{\theta}\log p_{\theta}(\mathrm{x_{t-1}|\mathrm{x}_t}) r(\mathrm{x}_{0})\bigg]
\end{equation}
% agregar objetivo DDPO_{IS}
\begin{equation}\label{eqn:ddpo-is-objective}
  (\text{DDPO}_{\text{IS}})~~ \nabla_{\theta}\mathcal{J} = \mathbb{E}_{\mathrm{x}_{T:0}\sim p_{\theta_{\text{old}}}} \bigg[\sum_{t=0}^{T}\frac{p_{\theta}(\mathrm{x}_{t-1}|\mathrm{x}_{t})}{p_{\theta_{\text{old}}}(\mathrm{x}_{t-1}|\mathrm{x}_{t})}\nabla_{\theta}\log p_{\theta}(\mathrm{x_{t-1}|\mathrm{x}_t}) r(\mathrm{x}_{0})\bigg]
\end{equation}
It is straightforward to compute $\log p_{\theta}$ considering that
$p_{\theta}$ is a conditional gaussian distribution 
$\mathcal{N}(f_{\theta}(\mathrm{x}_{t}) | \mathrm{x}_{t-1}, \alpha)$, where
$f_{\theta}$ is a neural network such as U-net architecture parameterized by $\theta$.

\section{Extending RL in diffusion models}

Simplifying the problem and ignoring the context raise from conditional generative models such as text-to-image, instead we assume an unconditional model free of context. Following...
\begin{equation}\label{difusion-rl-objective-2}
  \mathcal{J}_{\text{DDRL}}(\theta)
  = \mathbb{E}_{\mathrm{x}_{0}\sim p_{\theta}(\mathrm{x}_{T:0})}[R(\mathrm{x}_{T:0})]
\end{equation}

Based on the summarization of policy gradients in \cite{schulman2015high} we can see the design decision to extend reinforcement learning, these are non excluyentes...
\begin{equation}\label{eqn:general-pg-estimation-form}
  \nabla_{\theta}\mathcal{J}(\theta) = \mathbb{E}\bigg[\sum_{t=0}^{\infty}\Psi_{t}\nabla_{\theta}\log\pi_{\theta}(a_{t}|s_{t}) \bigg]
\end{equation}
La idea es que cualquier mejora de propuesta, utilice o no conocimientos
desde el campo de modelos de difusión, caiga en alguna de las siguientes
formulaciones para la optimización de policy gradient methods presentada
en el paper \textit{High-Dimensional Continuous Control Using Generalized Advantage Estimation}, aka GAE.

\subsection{Total Reward of the Trajectory}

\begin{equation}\label{eqn:psi-total-reward}
  \sum_{t=0}^{\infty}\Psi_{t} = \sum_{t=0}^{\infty} r_{t}
\end{equation}


\subsection{Reward following Action}

Reward on to go...

\begin{equation}\label{eqn:psi-reward-following-action}
  \sum_{t=0}^{\infty}\Psi_{t} = \sum_{t=t'}^{\infty} r_{t'}
\end{equation}

\subsection{Reward following Action with Baseline}

\begin{equation}\label{eqn:psi-reward-following-action-baseline}
  \sum_{t=0}^{\infty}\Psi_{t} = \sum_{t=t'}^{\infty} r_{t'} - b(s_{t})
\end{equation}


\subsection{MaDI: a masker to turn off non-informative pixels}

XYZ

\section{Empirical Analysis of Reward Trajectory Dynamics: Insights from DDPM samples}

¿Porqué valdría la pena extender el reward? ¿los estados intermedios 
aportan información?

% Reward signal during samples trajectories 
\begin{figure}[ht]
  \centering
  \begin{minipage}{0.5\textwidth}
      \centering
      \includegraphics[width=1\textwidth]{img/results/1k-trajectories-aestheic-score-single.png} % first figure itself
  \end{minipage}\hfill
  \begin{minipage}{0.5\textwidth}
      \centering
      \includegraphics[width=1\textwidth]{img/results/1k-trajectories-jpeg-size-single.png} % second figure itself
  \end{minipage}
  \vspace{-8pt}  % reduce space between caption and figure
    \captionsetup{width=\textwidth} % set the width of the caption
    \caption{\textbf{Visualizing reward signal during sample trajectories.} \textbf{Left:} Evolution of the aesthetic quality as reward signal over the six states $\mathrm{x}_{\tilde{t}}$, summarizing each of the $1000$ trajectories. \textbf{Right:} Image size after JPEG compression, providing another form of reward signal for the same set of trajectories.}
  \label{fig:samples-trajectory-rewards} % Add a proper reference for the label
\end{figure}


We utilize the \texttt{google/ddpm-celebahq-256} diffusion model \cite{ho2020denoising} to compile a dataset of trajectory samples, denoted as $\mathcal{S}_o$. This dataset comprises summaries of six intermediate states extracted from each trajectory. We specifically consider steps $t$ within the range of $0$ to $1000$, where the model was trained. To move from the final state of the diffusion chain, which is Gaussian noise, to the initial state, we leverage the reverse process of the diffusion chain.

Our methodology involves summarizing the sample trajectories by selecting intermediate steps at specific timestamps. We opted for a trajectory length of 40 steps, and interpolated within the original diffusion chain using the \texttt{DDIMScheduler} \cite{song2020denoising}. While obtaining the reward for the entire trajectory involves evaluating it over 40 times the original steps, we collect the rewards for every tenth step for memory and computational efficiency. Alternative approaches for selecting significant intermediate steps could be explored in future work.

Each state in the dataset is associated with a reward signal, which can be obtained using parameterized models such as the LAION aesthetic predictor or the ViT age classifier, or through functions like JPG compression size estimation.

\begin{enumerate}
\item The $\tilde{t}$ specified a subset of the step $t$ within the diffusion chain $t=0, \dots, T=1000$, in which the model \texttt{google/ddpm-celebahq-256} was trained.
\item Equivalence between $\tilde{t}$ and $t$ is given in the following tuples:
$(\tilde{t}=0, t=1000), (\tilde{t}=1, t=975), (\tilde{t}=2, t=725), (\tilde{t}=3, t=475), (\tilde{t}=4, t=225), (\tilde{t}=5, t=0)$.
\item The intermediate steps $\mathrm{x}_{t}, ~t\notin [0,1000]$, were collected every ten steps from the 40-step trajectory length. The reason was for memory and computation. Obtaining the reward for the complete trajectory requires moving from $6N$ to $40N$ reward evaluations.
\item Therefore, the dataset of samples $\mathcal{S}_{o} =  \big\{\mathrm{x}_{\tilde{t}}^{(i)}, \mathrm{r}_{\tilde{t}}^{(i)} \big\}_{\tilde{t}=0:5}~$, describe each $i$ trajectory from a total of $N$.
\end{enumerate}
%\item Every state has a $r\in \mathbb{R}$, corresponding to a reward. This signal can be obtained by a parameterized model such as \href{https://laion.ai/blog/laion-aesthetics/}{LAION aesthetic predictor} or \href{https://huggingface.co/nateraw/vit-age-classifier}{ViT age classifier}, or from functions such as obtaining the size of an image after JPG compression.

In Figure~\ref{fig:samples-trajectory-rewards} we can inspect...\\

Analyze the reward over predicted samples given an intermediate
state $\mathrm{x}_{t}$. \textit{Are intermediate samples with higher/lower reward as correlated with their predicted reward at the final sample?} We
will use the sample prediction $\tilde{\mathrm{x}}$ obtained by DDIM
\begin{equation}\label{ddim-predicted-sample}
  \tilde{\mathrm{x}}_{0}=\text{pred}(\mathrm{x}_{t})=\frac{\mathrm{x}_{t}-\sqrt{1-\alpha_{t}}\epsilon_{\theta}^{(t)}(\mathrm{x}_{t})}{\sqrt{\alpha_{t}}}
\end{equation}

¿Tiene lógica ver el \textit{reward} acumulado de una trayectoria completa?
¿No es un mal estimado el \textit{reward} un estado intermedio dado el ruido?

% No aporta muchoo el ruido
\begin{figure}[ht]
  \centering
  \includegraphics[scale=0.80]{img/results/samples-trajectories.png}
  \vspace{-18pt}  % reduce space between caption and figure
    \captionsetup{width=\textwidth} % set the width of the caption
    \caption{\textbf{DDPM samples trajectories.} Each row, from \textbf{top-to-bottom}, represents the best ($6.22$)
  and worst ($3.86$) aesthetic scores, and the highest ($34.38$) and lowest 
  ($4.07$) filesizes in kilobytes after JPEG compression for samples
  $\mathrm{x}_{0}$ in $\mathcal{S}_{o}$. Each column, from \textbf{left-to-right}, summarizes the
  states for each sample's trajectory. The rows correspond to the trajectories
  highlighted in the green and red lines of Figure~\ref{fig:samples-trajectory-rewards}}.
    \label{fig:sample-trajectories}
\end{figure}

\section{Model}

XYZ

\section{Experiments}

En primer lugar queremos reproducir la implementación propuesta en \textit{Training Diffusion Models with Reinforcement Learning} \cite{black2023training} en un \textit{setting} más simplificado para facilitar
la experimentación, pero que aún capture la complejidad de los text-to-image models para experimentar con \textit{reward functions}. Utilizaremos como
modelo preentrenado \texttt{google/ddpm-celebahq-256} descrito
en \cite{ho2020denoising}, el cual es un modelo no condicionado que
genera imágenes de dimensiones 256x256 de rostros humanos en RGB. Este modelo es previo a los modelos latentes de difusión, por lo que el proceso de \textit{denoising} ocurre en el espacio de los pixeles.


\subsection{LAION Aesthetic Score}

 XYZ

 \subsection{JPEG Compressibility \& Incompressibility}

 Using as a reward the size of the image after JPEG compression, we can 
 define two tasks: compressibility and incompressibility. For compressibility,
 we want to maximize the negative size of the image after compression. This
is equivalent to minimizing the size of the image after compression. 
The other side of the coin is to maximize the size of the image after compression, aka incompressibility.
  
 \subsection{Over 50 years old}

 XYZ

\begin{table}
\centering
\begin{tabular}{lccc}
\toprule
\textbf{Downstream Task} & \textbf{DDPM} & \textbf{DDPO} & \textbf{DDPOI}\\
\midrule
LAION Aesthetic Score ($[1,10]$) & 5.11 $~\pm$ 0.02 & \textbf{5.58} $~\pm$ 0.03 & - \\
JPEG compressibility (kb) & 17.26 $~\pm$  0.29 & \textbf{6.01} $~\pm$ 0.25  & -\\
JPEG incompressibility (kb) & 17.26 $~\pm$ 0.29 & \textbf{21.6} $~\pm$ 0.23 & - \\
Over 50 years old $P(x>\text{50})$ & 0.12 & - & - \\
\bottomrule
\end{tabular}
\captionsetup{width=\textwidth} % set the width of the caption
\caption{Mean and standard error of the mean (SEM) for rewards associated with each downstream task (rows) and each checkpoint trained based on the methods studied (columns). To ensure a fair comparison, all checkpoints were generated samples using the same initial noise. The \textbf{DDPM} column provides estimates of the initial capacities of the google/ddpm-celebahq-256 model, derived from the images in the dataset $\mathcal{D}_{o}$. The \textbf{DDPO} column corresponds to the checkpoint trained using DDPO with importance sampling, as referenced in \cite{black2023training}. The \textbf{DDPOI} column represents our proposed method. \textbf{TODO:} simplificar este caption mega verboso.}
\end{table}




% Mean and Reward Histograms for JPEG compressibility and incompressibility
\begin{figure}[ht]
  \centering
  \begin{minipage}{0.5\textwidth}
      \centering
      \includegraphics[width=1\textwidth]{img/results/reward_hist-laion-aesthetic.png} % first figure itself
      %\label{fig:sample_figure_1}
  \end{minipage}\hfill
  \begin{minipage}{0.5\textwidth}
      \centering
      % \includegraphics[width=1\textwidth]{img/results/reward_hist-jpeg-compressibility.png} % second figure itself
      %\label{fig:sample_figure_2}
  \end{minipage}\vspace{-0.1cm} % space between row 1 and row 2 of figures
  \begin{minipage}{0.5\textwidth}
      \centering
      \includegraphics[width=1\textwidth]{img/results/reward_hist-jpeg-compressibility.png} % third figure itself
      %\label{fig:sample_figure_3}
  \end{minipage}\hfill
  \begin{minipage}{0.5\textwidth}
      \centering
      \includegraphics[width=1\textwidth]{img/results/reward_hist-jpeg-incompressibility.png} % forth figure itself
      %\label{fig:sample_figure_4}
  \end{minipage}
  \vspace{-8pt}  % reduce space between caption and figure
    \captionsetup{width=\textwidth} % set the width of the caption
    \caption{\textbf{Learning curves from DDPO.} Evolution of the mean reward (black line) and histogram during the training steps for each \textit{downstream task}. The reward estimates were computed in each step using $100$ samples from the model.}
  \label{fig:reward_hist} % Add a proper reference for the label
\end{figure}

Maximizar LAION aesthetic score es un task mucho más complejo que JPEG Compressibility y JPEG Incompressibility (Por ver).

JPEG Incompressibility resulta más difícil de maximizar JPEG Compressibility
(Figure~\ref{fig:reward_hist}). Lo que tiene sentido según la naturaleza del
task y las capacidades del modelo. Hasta cierto punto, añadir mayor información 
requiere de mayores capacidades generativas para agregar semantica visual y 
otros features dentro de la imagen que no se pierdan durante el proceso de
compresión. Sin embargo, reducir el tamaño del archivo siempre se puede
alcanzar destruyendo la capacidad de generación del modelo, i.e. quitar
información es más fácil que agregar/crear información.

% Visual comparison between pretrained and ckpts fintuned with DDPO
\begin{figure}[ht]
  \centering
  \includegraphics[scale=0.72]{img/results/visual-comparison-results-200dpi.png}
  \vspace{-4pt}  % reduce space between caption and figure
    \captionsetup{width=\textwidth} % set the width of the caption
    \caption{\textbf{DDPO samples vs pretrained samples.} Qualitative depiction of the effects of RL finetuning on different reward functions.}
    \label{fig:visual-comparison-ddpo}
\end{figure}


\section{Discussion \& Limitations}

\lipsum[1]

\lipsum[4]



\chapter{Conclusion}

\ca{Esto completarlo al final, una sintesis de las discusiones finales del capítulo anterior y las limitaciones y perspectivas.}

\lipsum[1] 

\lipsum[2]


\typeout{}\bibliography{library}

% ------------------------------------------------------------------------------
% ANEXO
% Existe adicionalmente el entorno \begin{appendixd} que permite insertar
% \chapter y el entorno \begin{appendixdtitle}[style1] (4 estilos diferentes),
% el cual acepta \chapter y escribe el título de anexos encima
% ------------------------------------------------------------------------------
\begin{appendixs}
	
	\section{Additional DDPO samples}\label{appendix:additional-samples}

    % \textbf{Additional samples.} A hundred additional comparable samples from the pretrained and DDPO finetuned models are provided.

        % 100 samples from the pretrained model
        \begin{figure}[ht]
            \centering
            \includegraphics[scale=0.8]{img/results/ddpm-samples.png}
            \vspace{-4pt}  % reduce space between caption and figure
            \captionsetup{width=\textwidth} % set the width of the caption
            \caption{Celeba-HQ 256x256 generated samples from the pretrained model google/ddpm-celebahq-256.}
            \label{fig:ddpm-samples}
        \end{figure}

        % 100 samples from the finetuned model with ddpo and compressibility
        \begin{figure}
            \centering
            \includegraphics[scale=0.8]{img/results/ddpo-compressibility-samples.png}
            \vspace{-4pt}  % reduce space between caption and figure
            \captionsetup{width=\textwidth} % set the width of the caption
            \caption{Celeba-HQ 256x256 generated samples from the DDPO finetuned model optimized by JPEG compressibility.}
            \label{fig:ddpo-compressibility-samples}
        \end{figure}

        % 100 samples from the finetuned model with ddpo and incompressibility
        \begin{figure}
            \centering
            \includegraphics[scale=0.8]{img/results/ddpo-incompressibility-samples.png}
            \vspace{-4pt}  % reduce space between caption and figure
            \captionsetup{width=\textwidth} % set the width of the caption
            \caption{Celeba-HQ 256x256 generated samples from the DDPO finetuned model optimized by JPEG incompressibility.}
            \label{fig:ddpo-incompressibility-samples}
        \end{figure}

        % 100 samples from the finetuned model with ddpo and aesthetic quality
        \begin{figure}
            \centering
            \includegraphics[scale=0.8]{img/results/ddpo-aesthetic-samples.png}
            \vspace{-4pt}  % reduce space between caption and figure
            \captionsetup{width=\textwidth} % set the width of the caption
            \caption{Celeba-HQ 256x256 generated samples from the DDPO finetuned model optimized by aesthetic quality.}
            \label{fig:ddpo-aesthetic-samples}
        \end{figure}

    \newpage

    \section{Additional transitions from DDPM to DDPO}

        % transition using jpeg compressibility extra sample 1
        \begin{figure}
            \centering
            \includegraphics[scale=1.40]{img/results/compressibility_40.png}
            \vspace{-0pt}  % reduce space between caption and figure
            \captionsetup{width=\textwidth} % set the width of the caption
            \caption{\textbf{Additional example 1 of JPEG compressibility transformation during model updates}, starting from a pretrained DDPM model and optimized to maximize the reduction in image filesize after JPEG compression using DDPO.}
            \label{fig:ddpm-to-ddpo-compressibility-extra1}
        \end{figure}

        % transition using jpeg compressibility extra sample 2
        \begin{figure}
            \centering
            \includegraphics[scale=1.40]{img/results/compressibility_44.png}
            \vspace{-0pt}  % reduce space between caption and figure
            \captionsetup{width=\textwidth} % set the width of the caption
            \caption{\textbf{Additional example 2 of JPEG compressibility transformation during model updates}, starting from a pretrained DDPM model and optimized to maximize the reduction in image filesize after JPEG compression using DDPO.}
            \label{fig:ddpm-to-ddpo-compressibility-extra2}
        \end{figure}

        % transition using jpeg incompressibility extra sample 1
        \begin{figure}
            \centering
            \includegraphics[scale=1.40]{img/results/incompressibility_8.png}
            \vspace{-0pt}  % reduce space between caption and figure
            \captionsetup{width=\textwidth} % set the width of the caption
            \caption{\textbf{Additional example 1 of JPEG incompressibility transformation during model updates}, starting from a pretrained DDPM model and optimized to increase the image filesize after JPEG compression using DDPO.}
            \label{fig:ddpm-to-ddpo-incompressibility-extra1}
        \end{figure}


        % transition using jpeg incompressibility extra sample 2
        \begin{figure}
            \centering
            \includegraphics[scale=1.40]{img/results/incompressibility_28.png}
            \vspace{-0pt}  % reduce space between caption and figure
            \captionsetup{width=\textwidth} % set the width of the caption
            \caption{\textbf{Additional example 2 of JPEG incompressibility transformation during model updates}, starting from a pretrained DDPM model and optimized to increase the image filesize after JPEG compression using DDPO.}
            \label{fig:ddpm-to-ddpo-incompressibility-extra2}
        \end{figure}

        % transition using aesthetic quality extra sample 1
        \begin{figure}
            \centering
            \includegraphics[scale=1.40]{img/results/laion_1.png}
            \vspace{-0pt}  % reduce space between caption and figure
            \captionsetup{width=\textwidth} % set the width of the caption
            \caption{\textbf{Additional example 1 of aesthetic quality transformation during model updates}, starting from a pretrained DDPM model and optimized to maximize the LAION aesthetic score using DDPO.}
            \label{fig:ddpm-to-ddpo-aesthetic-extra1}
        \end{figure}


        % transition using aesthetic quality extra sample 2
        \begin{figure}
            \centering
            \includegraphics[scale=1.40]{img/results/laion_12.png}
            \vspace{-0pt}  % reduce space between caption and figure
            \captionsetup{width=\textwidth} % set the width of the caption
            \caption{\textbf{Additional example 2 of aesthetic quality transformation during model updates}, starting from a pretrained DDPM model and optimized to maximize the LAION aesthetic score using DDPO.}
            \label{fig:ddpm-to-ddpo-aesthetic-extra2}
        \end{figure}

        % transition from DDPM to DDPO samples optimized for Task.OVER50 (ViT age classifier) extra sample 1
        \begin{figure}
            \centering
            \includegraphics[scale=2.80]{img/results/over50_7.png}
            \vspace{-0pt}  % reduce space between caption and figure
            \captionsetup{width=\textwidth} % set the width of the caption
            \caption{\textbf{Additional example 1 of OVER50 transformation during model updates}, starting from a pretrained DDPM model and optimized to maximize the sum of logits for classes ≥ 50 years old of the ViT Age classifier using DDPO.}
            \label{fig:ddpm-to-ddpo-over50-extra1}
        \end{figure}

        % transition from DDPM to DDPO samples optimized for Task.OVER50 (ViT age classifier) extra sample 2
        \begin{figure}
            \centering
            \includegraphics[scale=2.80]{img/results/over50_46.png}
            \vspace{-0pt}  % reduce space between caption and figure
            \captionsetup{width=\textwidth} % set the width of the caption
            \caption{\textbf{Additional example 2 of OVER50 transformation during model updates}, starting from a pretrained DDPM model and optimized to maximize the sum of logits for classes ≥ 50 years old of the ViT Age classifier using DDPO.}
            \label{fig:ddpm-to-ddpo-over50-extra2}
        \end{figure}


    \newpage

    \section{Implementation details}\label{appendix:implementation}

    Each experiment conducted in this work is meticulously documented and accessible through their respective experiment dashboards on Weights \& Biases (W\&B), ensuring comprehensive logging of all relevant metrics and parameters. Additionally, the corresponding model checkpoints are stored and can be retrieved from Hugging Face model repositories, providing a seamless integration for further analysis and reproducibility. The details of these experiments, including the model checkpoints and experiment dashboards, are summarized in Table~\ref{tab:experiment_details}.

    \begin{table}[h!]
        \centering
        \begin{tabular}{l l l}
            \toprule
            \textbf{Experiment} & \textbf{Checkpoint (Hugging Face)} & \textbf{W\&B} \\
            \midrule
            DDPO/Aesthetic Quality & \href{https://huggingface.co/alkzar90/ddpo-aesthetic-celebahq-256}{ddpo-aesthetic-celebahq-256}  & \href{https://wandb.ai/alcazar90/ddpo-aesthetic-ddpm-celebahq256}{dashboard} \\
            DDPO/Compressibility & \href{https://huggingface.co/alkzar90/ddpo-compressibility-celebahq-256}{ddpo-compressibility-celebahq-256} & \href{https://wandb.ai/alcazar90/ddpo-compressibility-ddpm-celebahq256}{dashboard} \\
            DDPO/Incompressibility  & \href{https://huggingface.co/alkzar90/ddpo-incompressibility-celebahq-256}{ddpo-incompressibility-celebahq-256} & \href{https://wandb.ai/alcazar90/ddpo-incompressibility-ddpm-celebahq256}{dashboard} \\
            DDPO/OVER50 & NA & \href{https://wandb.ai/alcazar90/ddpo-over50-ddpm-celebahq256}{dashboard} \\
            \bottomrule
        \end{tabular}
        \caption{Experiment details with corresponding model checkpoints and experiment dashboards, including logging information.}
        \label{tab:experiment_details}
    \end{table}


    \subsection{Hyperparameters}

    XYZ

	% Tablas
	\enabletablerowcolor[2] % Activa el color de celda
	\begin{table}[H]
		\begin{threeparttable}
		\centering
		\caption{Hyperparameters used in the experiments.}
		\begin{tabular}{cccC{4cm}}
			\hline
			\textbf{Elemento} & $\epsilon_i$ & \textbf{Valor} & \textbf{Descripción} \bigstrut \\
			\hline
			A     & 10    & 3,14$\pi$ & Valor muy interesante\tnote{a} \\
			B     & 20    & 6 & Segundo elemento \\
			C     & 30    & 7 & Tercer elemento\tnote{1} \\
			D     & 150    & 10 & Sin descripción \\
			E     & 0    & 0 & Cero \\
			\hline
			\end{tabular}
		\begin{tablenotes}
			\item[a] Este elemento tiene una descripción debajo de la tabla
			\item[1] Más comentarios
		\end{tablenotes}
		\end{threeparttable}
		\label{tab:anexo-1}
	\end{table}
	\disabletablerowcolor % Desactiva el color de celda

    \subsection{Linear Warmup \& Half-Cosine Decay}

    \ca{\textbf{TODO:} decidir si vale la pena agregar esto...}

%     From the downstream task aesthetic quality the training dynamics present difficult to optimize the reward using a fixed learning rate. Improvement start to appears using...%The learning rate schedule is a linear warmup for the first 10\% of the training steps, followed by a half-cosine decay for the remaining 90\% of the training steps. The learning rate is multiplied by a factor of 0.1 at the end of the linear warmup and then decayed using a half-cosine decay schedule. The learning rate is initialized at $10^{-4}$ and the optimizer is Adam with $\beta_1 = 0.9$, $\beta_2 = 0.999$, and $\epsilon = 10^{-8}$.

%     Linear warmup followed by half cosine decay is a popular learning rate (LR) scheduling strategy used to fine-tune generative models like large language models (LLMs) and diffusion models. Here's an explanation of why this works and how it differs from using a fixed learning rate:

%     Why Linear Warmu0p with Half Cosine Decay Works
%     Stability During Initial Training: When training starts, the model's weights are usually initialized randomly or from a pre-trained state. A high learning rate at the beginning can cause instability and lead to large updates, which might push the model's weights into a poor local minimum. Linear warmup gradually increases the learning rate from a small value to a peak over a predefined number of steps, allowing the model to start training in a stable manner.

%     Efficient Exploration: The warmup phase helps the model explore the loss landscape more efficiently by starting cautiously. This ensures that the initial updates do not disrupt the pre-trained weights drastically.

%     Gradual Refinement: After the warmup phase, the learning rate follows a half cosine decay schedule, which gradually reduces the learning rate from its peak to a minimum value. This slow reduction allows the model to fine-tune and converge more smoothly, as the smaller learning rates towards the end of training help in fine-tuning the model parameters precisely without large oscillations.

%     Prevents Overfitting: By gradually reducing the learning rate, the model's updates become smaller over time, which can help prevent overfitting. The smaller updates towards the end of training allow the model to settle into a more optimal solution.

%     Alignment with Human Learning Patterns: Some studies suggest that the cosine decay mimics human learning patterns, where initial learning is fast and exploratory, but as expertise develops, the learning rate naturally slows down for fine-tuning and mastery.

%     Difference from Fixed Learning Rate
%     Initial Instability: With a fixed high learning rate, training can be unstable at the beginning, leading to erratic updates and potential divergence. A fixed low learning rate, on the other hand, can make the training process extremely slow, especially in the initial stages where large updates might be necessary to escape poor local minima.

%     Lack of Gradual Refinement: Fixed learning rates do not provide the gradual refinement that cosine decay offers. This can result in suboptimal convergence, as the model might not settle into the optimal solution smoothly.

%     Potential Overfitting: A fixed high learning rate throughout the training can lead to overfitting, as the model continues to make large updates even when it's close to convergence. Conversely, a fixed low learning rate might result in underfitting, as the model might not learn the intricate patterns in the data effectively.

%     References to Literature
%     "Attention Is All You Need" (Vaswani et al., 2017): Introduced the concept of warmup for learning rates in the context of training the Transformer model. The paper discusses the importance of gradually increasing the learning rate to stabilize training.

%     "SGDR: Stochastic Gradient Descent with Warm Restarts" (Loshchilov \& Hutter, 2017): Discusses the use of cosine annealing for learning rates, showing how it can improve the training of deep neural networks by periodically decaying and restarting the learning rate.

%     "BERT: Pre-training of Deep Bidirectional Transformers for Language Understanding" (Devlin et al., 2019): Utilizes a learning rate schedule with warmup and linear decay, highlighting the benefits of such schedules in fine-tuning large models effectively.

%     "Decoupled Weight Decay Regularization" (Loshchilov \& Hutter, 2019): Proposes AdamW optimizer and discusses the importance of proper learning rate scheduling, including warmup and decay strategies.

%     These papers provide a foundation for understanding the benefits of learning rate schedules involving warmup and decay, emphasizing their role in stabilizing and optimizing the training of complex generative models.



\end{appendixs}

% FIN DEL DOCUMENTO
\end{document}